\documentclass[a4paper,11pt]{article}

\usepackage[utf8]{inputenc}
\usepackage[czech]{babel}
\usepackage[left=2cm,top=3cm,text={17cm,24cm}]{geometry}
\usepackage{graphicx}
\usepackage{listings}
\usepackage{url}
\usepackage{graphicx}
\graphicspath{ {./images/} }
\title{ISA - Laboratorní cvičení č.\,2\\
{\bf\large Zabezpečený přenos dat}}

\author{Vysoké učení technické v Brně}

\date{\url{https://github.com/nesfit/ISA/tree/master/encrypted_transfers}}

\begin{document}

{\let\newpage\relax\maketitle}

\section*{Cíl laboratorního cvičení:}
\begin{itemize}
  \item Základní seznámení se synchronizací času a protokolem NTP.
  \item Naučit se práci s nástrojem SSH a správu klíčů.
  \item Naučit se základy protokolu TLS.
  \item Seznámit se s Certificate Transparency Log.
\end{itemize}

\section*{Pokyny}
\begin{itemize}
  \item Přihlaste se do OS GNU/Linux (F3), user/password {\tt user}/{\tt user4lab}.
  \item V případě potřeby se přepněte na uživatele {\tt root} příkazem {\tt su}
  (switch user), heslo {\tt root4lab}.
  \item V případě potřeby si otevřete další terminál v novém okně.
  \item Pro editaci konfiguračních souborů použijte libovolný editor (např.
  nano, mcedit, vim, gedit).
\item Pracujte ve dvojicích (váš počítač je dále v textu označen jako hXX a
  počítač souseda hYY, kde XX a YY je číslo počítače 01--20).
  %Zkontrolujte, že máte počítač propojen
  % přímým kabelem s přístupovým přepínačem (rozhraní enp2s0, zdířka E na patch
  %panelu).
  Pokud souseda nemáte, zapněte si jeden z volných počítačů.

\end{itemize}

%\newpage
%\section{Laboratorní úlohy}
\section{NTP}

Vašim úkolem je zajistit synchronizaci hodin počítače. V rámci bezpečnosti je
důležité provozovat počítač se správným časem kvůli časovým razítkům
označujícím platnost a neplatnost klíčů, certifikátů, podpisů apod.

\begin{enumerate}
  \item Příkazem {\tt systemctl status chronyd.service} zjistěte, zda běží služba
    \texttt{chronyd} pro synchronizaci času pomocí protokolu NTP.

  \item Do protokolu vyplňte adresu nakonfigurovaného \textbf{NTP serveru} a jeho \textbf{hodnotu stratum}.

  \item \textit{(volitelně)} Pod uživatelem \texttt{root} Otevřete
    konfiguračního soubor {\tt /etc/chrony.conf}. V souboru
    nahraďte využívání poolu serveru Cent OS ({\tt *.centos.pool.ntp.org}) za
    server na FIT ({\tt ntp.fit.vutbr.cz}).

    Po úpravě restartujte démona pro NTP ({\tt systemctl restart chronyd.service}).
    Ověřte pomocí {\tt systemctl status chronyd.service}, že služba běží; pokud
    neběží, zobrazte chyby pomocí příkazu {\tt journalctl -u chronyd}, nalezené
    chyby v konfiguraci opravte a službu {\tt chronyd} restartujte.
    Zadejte znovu {\tt chronyc -n sources} a zkontrolujte, zda je čas nyní sysnchronizován se serverem FIT.

\end{enumerate}

\section{Vzdálený terminál -- SSH, Secure Shell}

%\noindent {\bf V průběhu řešení jsou tučně vyznačené otázky, na které si připravte odpovědi.
%  Na konci úkolu je sdělíte cvičícímu.}
%~\\~\\
Namísto řetězce {\tt <login>} používejte své studentské přihlašovací jméno.

\begin{enumerate}

  \item Otevřete si dvě okna: příkazovou řádku pro uživatele {\tt user} a další
    příkazovou řádku pro uživatele {\tt root} příkazem {\tt su} (switch user).
Pomocí příkazu {\tt whoami} vypište jméno aktivního uživatele, ověřte, že je
    očekávané.

    V~obou otevřených terminálech dočasně vypněte podporu pro SSH agenta:
  \begin{lstlisting}
  [user@hXX]$ unset SSH_AUTH_SOCK
  [root@hXX]# unset SSH_AUTH_SOCK
  \end{lstlisting}

  Pokud budete otevírat v průběhu řešení úkolu nové terminály, nezapomeňte
  v nich také vypnout podporu agenta SSH.

  \item {\bf Bezpečné připojení na vzdálený počítač bez autentizačních klíčů.}

    \begin{enumerate}

      \item Spusťte program Wireshark a zachytávejte komunikaci na portu 22
        (rozhraní enp2s0).

      \item Přihlaste se na počítač souseda hYY příkazem {\tt ssh user@hYY.netlab.fit.vutbr.cz.} a zadejte heslo \textbf{user4lab}.

      \item Na serveru zadejte libovolný příkaz, který znáte (např. zobrazte
        obsah manuálové stránky ssh příkazem {\tt man ssh}, vypište
        obsah adresáře příkazem {\tt ls}, obsah souborů příkazem {\tt ls},
        aktuálního uživatele příkazem {\tt whoami} apod.).

      \item Příkazem {\tt exit} nebo stiskem {\tt Ctrl-D} spojení ukončete.

      \item V programu Wireshark zobrazte obsah komunikace (pomocí Follow TCP stream).
        Zhodnoťte, \textbf{co lze z uvedené komunikace vyčíst}. Jsou v komunikaci vidět
        \textbf{zadávané příkazy a jejich výstup?} Vaše odpovědi zapište do protokolu.


    \end{enumerate}

  \item {\bf Vytvoření veřejného a privátního klíče.}

    \begin{enumerate}

      \item Jako uživatel {\tt user} vygenerujte příkazem \verb|$ ssh-keygen -C <login>@user|
        implicitní klíč pro uživatele
        {\tt user}. Neměňte jeho
        název a zvolte heslo o délce alespoň osmi znaků, například
        \texttt{fitvutisa}.

      \item Jako uživatel {\tt root} vygenerujte příkazem \verb|# ssh-keygen -N "" -C <login>@root|
        implicitní klíč pro uživatele {\tt root} bez hesla.

      \item Ověřte obsah a přístupová práva u nově vzniklých souborů (\verb|ls -l ~/.ssh|).

      \item Do protokolu uveďte, jaký je \textbf{význam obou klíčů}, v \textbf{jakých souborech} se nacházejí a~jaké mají nastaveno \textbf{oprávnění}.


    \end{enumerate}

  \item {\bf Distribuce klíčů}

    \begin{enumerate}

      \item Oba veřejné klíče zkopírujte na vzdálený počítač hYY do
        souboru \verb|.ssh/authorized_keys|
        pro klíč uživatele {\tt user} např.: \\
        {\verb&cat ~user/.ssh/*.pub | ssh user@hXX "cat >> .ssh/authorized_keys"&}\\
        a nastavte správná přístupová práva:\\
        {\verb&ssh user@hXX "chmod g-w .ssh/authorized_keys"&} \\
        pro klíč uživatele {\tt root} např.: \\
        {\verb&cat /root/.ssh/*.pub | ssh root@hXX "cat >> .ssh/authorized_keys"&}. \\

      \item Zkuste se znovu přihlásit jako \texttt{user} i jako \texttt{root} na stejný vzdálený počítač hYY.

      \item Do tabulky v~protokolu vyplňte, \textbf{jaká hesla bylo nutné zadat}. Vysvětlete také, jaký
      je význam souboru \texttt{authorized\_keys}.

      \item Zkuste zadat \textbf{opakovaně špatné heslo}. Do protokolu napište, co se stalo.
      Při experimentech můžete také využít tzv. verbose režim ssh (\texttt{ssh -v}).

    \end{enumerate}

  \item {\bf Omezení použití klíčů}

    Nyní bude naším cílem omezit použití klíče uživatele {\tt root}, který není chráněn heslem tak,
    aby pomocí něj bylo možné na vzdáleném serveru vykonat pouze konkrétní příkaz.

    \begin{enumerate}

      \item Přihlaste se jako uživatel {\tt root} na počítač souseda hYY, kam jste nakopírovali své veřejné klíče. Na počítači souseda upravte
        upravte soubor s autorizovanými veřejnými klíči tak, že na začátek
        řádku s klíčem uživatele {\tt root} (řádek poznáte tak, že končí řetězcem
        {\tt <login>@root}) napíšete
        \verb|command="chronyc -n sources" | \\ (následovaný jednou mezerou a původním
        obsahem řádku).
        Odhlaste se ze vzdáleného počítače.

      \item Znovu se přihlaste na počítač souseda hYY z
        účtu \texttt{root} jako \texttt{root}.  \textbf{Aplikovalo se
        omezené využití klíče? Pokud ano, jak se projevilo?} Odpovězte v~protokolu.

    \end{enumerate}


  \item {\bf Pohodlné opakované použití klíče zabezpečeného heslem.}

    \begin{enumerate}

      \item Ukončete terminál uživatele {\tt user} a vytvořte nový. Pomocí
        příkazu \verb|env| ověřte, že je nastavená proměnná
        \verb|SSH_AUTH_SOCK|.

      \item Přihlaste se znovu k počítači hYY.

      \item Odhlaste se a přihlašte ještě jednou. \textbf{Bylo tentokrát nutné znovu zadávat heslo?}
      Odpovězte v~protokolu.


    \end{enumerate}


\end{enumerate}

\section{Zabezpečení transportní vrsvy -- TLS, Transport Layer Security}

{\bf V průběhu řešení jsou tučně vyznačené otázky, na které si připravte odpovědi.
  Na konci úkolu je sdělíte cvičícímu.}

\begin{enumerate}

  \item {\bf  Nezabezpečený přenos dat}

    \begin{enumerate}

      \item Spusťte program Wireshark, zachytávejte komunikaci na portu 80.

      \item Pomocí programu {\tt telnet} se připojte k fakultnímu webovému
        serveru: \\ \verb|telnet nes.fit.vutbr.cz 80|.

      \item V programu Wireshark pozorujte navázání spojení TCP pomocí
        trojcestného handshaku.

      \item Zašlete serveru požadavek protokolem HTTP, např.:
        \verb|GET / HTTP/1.0|, dotaz ukončete prázdným řádkem. V terminálu pozorujte
        odpověď.

      \item Zobrazte si v programu Wireshark komunikaci pomocí HTTP.
      Do~protokolu uveďte, zda a~proč je/není možné \textbf{přečíst obsah komunikace}.

    \end{enumerate}

  \item {\bf Přenos dat zabezpečený TLS}

    \begin{enumerate}

      \item Spusťte program Wireshark, zachytávejte komunikaci na portu 443.

      \item Pomocí programu {\tt openssl} se připojte k fakultnímu webovému
        serveru: \\ \verb|openssl s_client -connect www.fit.vutbr.cz:443 -tls1_2|.
        Všimněte si řádku \emph{Verify return code}, co
        říká o certifikátu?

      \item V programu Wireshark pozorujte navázání spojení TCP a TLS. Je možné zjistit \textbf{jméno serveru}?
      Zodpovězte v protokolu.

      \item Pomocí programu \texttt{openssl} dále v rámci navázaného spojení
        zašlete serveru požadavek protokolem HTTP, např.:
        \verb|GET / HTTP/1.0|, dotaz ukončete prázdným řádkem. V terminálu pozorujte
        odpověď.

      \item Zobrazte si tuto komunikaci programu Wireshark.
      Do~protokolu uveďte, zda a~proč je/není možné \textbf{přečíst obsah komunikace}.

      \item Z výstupu aplikace \texttt{openssl} určete, jaká \textbf{šifrovací sada (cipher suite)} se používá.
      Její identifikátor zapište do protokolu.

      \item Navštivte stránku \url{https://ciphersuite.info/}. Najděte si detaily k dané
      šifrovací sadě. V protokolu \textbf{vyplňte do tabulky požadované informace}.
      (Pozn. Můžete také využít informací z nástroje Wireshark

    \end{enumerate}

  \item {\bf Zabezpečený přenos dat TLS v prohlížeči}

    \begin{enumerate}

      %\item Spusťte program Wireshark, zachytávejte komunikaci na portu 443.

      \item Spusťte prohlížeč Firefox a otevřete v něm stránku
        \verb|wis.fit.vutbr.cz|.

      \item Klikněte na ikonku zámku nalevo od URL a~najděte, jaká \textbf{šifrovací
      sada} je použita zde. Zobrazte si také \textbf{informace o použitém certifikátu}.
      Požadované údaje vyplňte do protokolu.

      \item V menu \emph{Edit » Settings » Privacy \& Security} v sekci
        \emph{Certificates} zobrazte certifikáty autorit, kterým důvěřujete.
        Odhadněte, kolik jich zhruba je. \textbf{Názvy několika} (2-3) z nich zapište do protokolu.

      %\item V programu Wireshark pozorujte navázání spojení TCP a TLS. {\bf Jaké
      %  informace lze vyčíst z~handshaku TLS? Je možné zjistit jméno serveru?
      %  Nalezené jméno serveru na konci úkolu ukažte cvičícímu}.

    \end{enumerate}

  \item {\bf Certificate Transparency}

    \begin{enumerate}

      \item Příkazem \verb|host -t CAA fit.vutbr.cz| si zobrazte certifikační
        autority oprávněné vydávat certifikáty pro doménu \url{fit.vutbr.cz}.

      \item V prohlížeči se připojte na stránku
        \url{https://certstream.calidog.io/}, klikněte na tlačítko \emph{Open
        Fire Hose}. Pozorujte, \textbf{jaké informace} se zde zobrazují.
        Zamyslete se, k čemu mohou být tyto informace \textbf{dobré} a jestli jich není
        možné nějak \textbf{zneužít}. Odpovězte v protokolu.

      \item V prohlížeči se připojte na stránku \url{https://crt.sh}.

      \item Nalezněte certifikáty vydané pro servery v rámci domény
        \emph{kralovopole.brno.cz}.

      \item Najděte na stránce ikonu směřují k souboru \textbf{Atom}.
      Podívejte se, co soubor obsahuje a zodpovězte otázku v protokolu.
      Zamyslete se, k čemu by se takový obsah dal použít.
    \end{enumerate}

\end{enumerate}


\section{Ukončení práce v laboratoři}
\begin{itemize}
  \item Jako uživatel \texttt{root} vypněte počítač dávkou {\tt /root/isa2/clean}.
\end{itemize}

\end{document}
%% END OF FILE

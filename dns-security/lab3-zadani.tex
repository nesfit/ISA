\section*{Cíl laboratorního cvičení}
\begin{itemize}
  \item Seznámit se s systémem DNS.
  \item Prozkoumat data přenášená v protokolu DNS, DNS over HTTPS pomocí programu Wireshark.
  \item Rozšifrovat zachycenou komunikaci z prohlížeče (Firefox).
  \item Nastavit šifrovanou komunikaci DNS over TLS.
  \item Nastavit šifrovanou komunikaci DNS přes DNS over TLS a zároveň filtrace DNS reklamních a malware domén.
\end{itemize}

\section*{Pokyny}
\begin{itemize}
  \item Pro práci v cvičení budeme používat virtuální stroj v programu
  VirtualBox\footnote{\url{https://nes.fit.vutbr.cz/isa/ISA2020.ova}}. Pokud z předchozích cvičení máte provedeny nějaké změny ve virtuálním stroji, obnovte ho do výchozího stavu.
  \item Před zahájením cvičení si vytvořte snapshot virtuálního stroje za pomoci menu \textit{Machine $\rightarrow$ Take snapshot} pro snadný návrat k výchozímu stavu.
  \item Odpovědi pište do odpovědního archu \texttt{protokol.md} který odevzdáte do WIS-u. Dostupný je na adrese \url{https://github.com/nesfit/ISA/blob/master/dns-security/protokol.md}.
  \item Do WIS-u budete také odevzdávat všechny zachycené \texttt{pcap} soubory.
  \item Uživatelé a hesla pro přihlášení: \texttt{user - user4lab}, \texttt{root - root4lab}.
  \item Přihlaste se jako uživatel \texttt{user}. Veškeré potřebné příkazy následně spouštějte jako \texttt{root}.
\end{itemize}


\section{Resolving DNS dotazů}
\begin{enumerate}
    \item Spusťte program Wireshark (vždy jako \texttt{root} z příkazové řádky příkazem \texttt{wireshark \&}) a začněte zachytávat komunikaci na rozhraní, pomocí kterého jste připojeni k Internetu (\texttt{enp0s3}).
    \item Otevřete terminál a pomocí příkazu \texttt{nslookup -type=ns vutbr.cz} zjistěte autoritativní DNS servery pro doménu \texttt{vutbr.cz} a zapište je do odpovědního archu.
    \item Zastavte zachytávání komunikace v programu Wireshark. Zachycený provoz uložte do souboru \texttt{cv3-dns.pcap}, který budete odevzdávat.
    \item V zachyceném provozu nalezněte pakety obsahující komunikaci Vámi provedeného dotazu na doménu \texttt{vutbr.cz} a prozkoumejte je.
	\item Do odpovědního archu zapište \emph{display filter}, kterým vyfiltrujete pouze pakety související s DNS provozem.
	\item Kolik paketů souvisejících s Vaším dotazem na doménu bylo zachyceno? Číslo zapište do odpovědního archu.
	\item Byl proveden rekurzivní nebo iterativní DNS dotaz? Jak jste to zjistili ze zachyceného provozu? Zapište do odpovědního archu.
\end{enumerate}

\section{Zabezpečení a resolving pomocí DNS over HTTPS}
\begin{enumerate}
    \item Spusťe prohlížeč Firefox.
    \item V prohlížeči přistupte do \texttt{Preferences} pak sescrollujte dolů na položku \texttt{Network Settings} a klikněte na tlačítko \texttt{Settings}. V dialogovém okně najděte položku \texttt{Enable DNS over HTTPS} a zaškrtněte. Provider nastavte na \texttt{Custom} a vyplňte nově vzniklé pole tímto url\\ \texttt{https://odvr.nic.cz/doh}.
    \item Potvrďte změny kliknutím na tlačítko \texttt{OK} a prohlížeč zavřete.
    \item Spusťte program Wireshark jako uživatel \texttt{root} z terminálu příkazem \texttt{wireshark \&} a začněte zachytávat provoz na všech rozhraních.
    \item Pro pozdější rozšifrování HTTPS komunikace z prohlížeče je nutné nastavit proměnnou prostředí\\ \texttt{SSLKEYLOGFILE=<cesta\_k\_souboru>}, na kterou prohlížeče Firefox, Chrome a případně další reagují.
    \item Otevřete terminál a jako uživatel \texttt{user} zadejte\\ \texttt{export SSLKEYLOGFILE=/home/user/Desktop/keylogfile.log}.
    \item Následně ve stejném okně terminálu, ve kterém jste nastavili proměnnou prostředí \texttt{SSLKEYLOGFILE}, spusťte jako uživatel \texttt{user} prohlížeč Firefox příkazem \texttt{firefox \&}.
    \item Přistupte na pár webových stránek, které Vás napadnou. Následně prohlížeč zavřete a tuto akci ještě jednou či vícekrát zopakujte, pokaždé ideálně s jinými webovými stránkami. Nakonec prohlížeč zavřete.
    \item Následně zastavte zachytávání v programu Wireshark.
    \item Zachycenou komunikaci uložte do souboru \texttt{cv3-DoH.pcap}, který budete odevzdávat.
	\item Představ si sebe jako útočníka, který zachytil neznámou komunikaci do souboru \texttt{cv3-DoH.pcap} a nemá k ní žádné další informace. Dokázali byste v tuto chvíli zjistit ze zachyceného DNS provozu, jaké domény byly přes prohlížeč Firefox navštíveny? Proč? Odvěď uveďte do odpovědního archu.
    \item V programu Wireshark otevřete \texttt{Edit > Preferences} a zde v levém sloupci rozklikněte \texttt{Protocols} a zde nalezněte položku \texttt{TLS}. Na této kartě je potřeba nastavit \texttt{(Pre)-Master-Secret log filename} na váš keylogfile (\texttt{/home/user/Desktop/keylogfile.log}). Aplikujte změnu kliknutím na \texttt{OK}. Nyní by mělo proběhnout rozšifrování provozu.
    \item Pomocí \emph{display filteru} vyfiltrujte pouze TLS provoz. Do odpovědního archu zapište, jaký \emph{display filter} jste zadali. Pokuste se nalézt pakety protokolu DoH (DNS over HTTPS).
    \item Pokud se Vám to podařilo, podívejte se jak vypadá obsah paketu.
    \begin{enumerate}
        \item Pokud se vám nepodařilo nalézt pakety s DoH, ve složce \texttt{/home/user/doh-pcaps/} naleznete \texttt{.zip} soubor po jehož rozbalení objevíte soubor \texttt{doh-decrypted.pcapng}.
        \item Když ve Wiresharku otevřete tento soubor, měli byste narazit na již dešifrovaný TLS provoz, ve kterém DoH pakety již určitě naleznete.
    \end{enumerate}
	\item Vyberte si libovolnou zachycenou DoH odpověď a do odpovědního archu opište jeden řádek z položky \texttt{Answers}.
	\item Jaká je cílová IP adresa pro pakety s DoH dotazy? Jaké doménové jméno patří k této IP adrese? Zapište do odpovědního archu.
    \item Před postupem k další části cvičení nezapomeňte otevřít prohlížeč a na stejném místě v \texttt{Preferences} položku \texttt{Enable DNS over HTTPS} opět vypnout a prohlížeč zavřít.
\end{enumerate}


\section{Zabezpečení a resolving pomocí DNS over TLS}
\label{sec:dot}
\begin{enumerate}
    \item Deaktivujte selinux jako uživatel \texttt{root} pomocí příkazu \texttt{setenforce 0}.
	\item Zároveň v souboru \texttt{/etc/selinux/config} zkontrolujte a případně upravte řádek s proměnnou \texttt{SELINUX} následovně \texttt{SELINUX=disabled}. NErestartujte počítač.
    \item Nainstalujete Unbound DNS caching resolver pomocí příkazu \texttt{yum install unbound -y}.
    \item V souboru \texttt{/etc/unbound/unbound.conf} najděte příslušnou část pro úpravu forward zón (pod řádkem začínajícím \texttt{\# Forward zones} a přidejte následující:
    
\begin{verbatim}
forward-zone:
    name: "."
    forward-ssl-upstream: yes
    # Cloudflare DNS
    forward-addr: 1.1.1.1@853
    forward-addr: 1.0.0.1@853
\end{verbatim}

    \item Jakmile je soubor upraven, uložte jej a restartujte službu pomocí příkazu \texttt{systemctl restart unbound}.
    \item Ověřte zdali služba běží bez problémů pomocí \texttt{systemctl status unbound}.
    \item V souboru \texttt{/etc/resolv.conf} nastavte nameserver na IP adresu \texttt{127.0.0.1}.
    \item Nyní spusťte program Wireshark a spusťte zachytávání na rozhraní, pomocí kterého jste připojeni k Internetu.
    \item Pokuste se vygenerovat nějaký DNS provoz pomocí webového prohlížeče, a to konkrétně přístupem na web \texttt{idnes.cz}.
    \item Vyčkejte než se stránka celá načte a důkladně si ji prohlédněte.
    \item Následně zavřete tab s načtenou stránkou i prohlížeč samotný.
    \item Zastavte zachytávání provozu. Zachycený provoz uložte jako soubor \texttt{cv3-DoT.pcap}, který budete odevzdávat.
    \item V programu Wireshark pomocí \emph{display filteru} vyfiltrujte pouze pakety, které využívají port 853 nad protokolem TCP. Jaký filtr přesně jste použili? Zapište do odpovědního archu.
    \item Následně vyfiltrujte pakety, které obsahují port 53 nad protokolem TCP nebo UDP. Jaký filtr přesně jste použili zde? Opět zapište do odpovědního archu.
	\item Jaká služba běží nad portem 53? Kolik paketů se zdrojovým nebo cílovým portem 53 bylo zachyceno? Odpovězte do odpovědního archu a zamyslete se, proč právě takové číslo.
\end{enumerate}


\section{Blokování reklam a další}
\begin{enumerate}
    \item Na začátku souboru \texttt{/etc/unbound/unbound.conf} pod řádek obsahující \texttt{server:} přidejte následující:
\begin{verbatim}
    interface: 127.0.0.1
    port: 5335
    do-ip4: yes
    do-udp: yes
    do-tcp: yes
    do-ip6: yes
\end{verbatim}
    \item Restartujte službu unbound DNS caching serveru pomocí příkazu \texttt{systemctl restart unbound}.
    \item Pomocí příkazu \texttt{systemctl enable unbound} nastavte povolení služby na systému.
    \item V souboru \texttt{/etc/resolv.conf} nastavte nameserver dočasně na IP adresu \texttt{8.8.8.8}. Dohledejte, jaké doménové jméno náleží k IP adrese 8.8.8.8 a jakou hlavní službu servery s touto IP adresou poskytují? Zapište do odpovědního archu.
    \item Nainstalujte pi-hole pomocí příkazu \texttt{curl -sSL https://install.pi-hole.net | bash}.
    \item Instalací pi-hole vás bude provázet dialogové okno, ve kterém bude nutné vybrat následující možnosti:
    \begin{enumerate}
        \item V dialogovém okně budete dotázáni na instalaci PHP. V tomto případě vyberte možnost \texttt{no}.
		\item Potvrďte, že souhlasíte, aby se Vaše zařízení stalo blokátorem reklam.
		\item Potvrďte informaci, že se jedná o software zdarma.
		\item Potvrďte, že rozumíte nutnosti přidělení statické IP adresy v lokální síti.
        \item Při výběru síťového rozhraní nechte zaškrtnuté ethernetové (\texttt{enp0s3}).
        \item U výběru "Upstream DNS Provider" sescrollujte dolů, kde zvolte možnost \texttt{Custom} a následně pro adresu serveru zadejte \texttt{127.0.0.1\#5335}. A v dalším okně potvrďte.
        \item Dále při výběrů blocking listů ponechte zaškrtnuté oba dva listy.
        \item Zbytek nastavení ponecháme ve výchozím stavu a jenom vždy odsouhlasíme buď \texttt{yes} nebo \texttt{ok}.
    \end{enumerate}
    \item Jakmile je instalace pi-hole dokončena, restartujte systém pomocí příkazu \texttt{reboot}.
    \item Přihlašte se do systému a zkontrolujte zdali pi-hole běží v pořádku:
    \begin{enumerate}
        \item Spusťte příkaz \texttt{pihole status}.
        \item Zkontrolujte obsah souboru \texttt{/etc/resolv.conf}, měl by obsahovat záznam pro nameserver ukazující na \texttt{127.0.0.1\#5335}.
    \end{enumerate}
    \item Spusťte prohlížeč Firefox a přistupte znovu na stránku \texttt{idnes.cz}. Jaký rozdíl jste vypozorovali? Zapište do odpovědního archu.
	\item Pořiďte snímek obrazovky s načtenou webovou stránkou \texttt{idnes.cz} tak, aby vypozorovaný rozdíl byl na snímku obrazovky viditelný, a uložte snímek obrazovky jako obrázek s názvem\\ \texttt{cv3-idnes.\{jpg|jpeg|tiff|...\}} v dostatečně nízkém rozlišením, aby ho bylo možné odevzdat do WIS-u.
\end{enumerate}

\section*{Poznámka ke cvičení}
Použití nasazení nástroje pi-hole lokálně na zařízení není úplně standardní a je v této konfiguraci pouze z demonstračních důvodů. Standardně je tato aplikace určena pro nasazení na samostatném serveru (například Raspberry Pi) běžícím v lokální síti a použití tohoto serveru všemi zařízeními v síti jednotně. Je možné jej pak zkombinovat i s tunelováním DNS přes TLS jako v tomto cvičení (viz úloha \ref{sec:dot}). Pro případné nasazení na serveru v lokální síti bude potřeba pár drobných změn v tomto cvičení.

\section*{Odevzdávané soubory}
Zkontrolujte, zda máte všechny soubory které se budou odevzdávat:
\begin{itemize}
  \item \texttt{protokol.md}
  \item \texttt{cv3-dns.pcap}
  \item \texttt{cv3-DoH.pcap}
  \item \texttt{cv3-DoT.pcap}
  \item \texttt{cv3-idnes.\{jpg|jpeg|tiff|...\}}
\end{itemize}

\section*{Ukončení práce v laboratoři}
\begin{itemize}
	\item Do WIS-u odevzdejte vyplněný \texttt{protokol.md}, všechny zachycené \texttt{pcap} soubory a snímek obrazovky.
	\item Vypňete virtuální stroj a obnovte jeho snapshot vytvořený na začátku této laboratoře.
\end{itemize}

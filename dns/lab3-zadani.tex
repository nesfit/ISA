
\section*{Cíle laboratoře}
\begin{itemize}
  \item Seznámení se systémem DNS a databází Whois.
  \item Prozkoumání dat přenášených v~protokolu DNS pomocí programu Wireshark.
  \item Blokování vybraných doménových jmen.
  \item Konfigurace a spuštění DNS serveru obsluhujícího vlastní doménu.
\end{itemize}

\section*{Základní instrukce}
\begin{itemize}
  \item Přihlaste se do OS CentOS (F3), user/password {\tt user}/{\tt user4lab}.
  \item Otevřete si příkazovou řádku pro uživatele {\tt user}.
  \item Otevřete si příkazovou řádku pro uživatele {\tt root} příkazem {\tt su}
    (switch user).
  \item Pro editaci konfiguračních souborů použijte libovolný editor (např.
    nano, vim, gedit).
\end{itemize}

\section*{Úkoly}

\section{Seznámení s~DNS}
\begin{enumerate}
    \item Spusťte program Wireshark (vždy jako \texttt{root} z~příkazové řádky příkazem \texttt{wireshark \&}) a začněte zachytávat DNS komunikaci na rozhraní, pomocí kterého jste připojeni k~Internetu (\texttt{enp2s0}).
    \item Otevřete terminál a pomocí příkazu \texttt{nslookup -type=a vutbr.cz} zjistěte, na jakou IPv4 adresu se překládá doména \texttt{vutbr.cz}.
    \item Zachycený DNS dotaz a odpověď si prohlédněte v~programu Wireshark.
    \item Příkazem \texttt{nslookup -type=soa vutbr.cz} si zobrazte SOA záznam naší Alma mater\footnote{\textbf{Alma mater} (latinsky \textbf{matka živitelka}) je původně antické označení pro bohyni matku. Ve středověké poezii se spojení někdy užívalo i pro Pannu Marii jako \emph{Matku Boží} (např. v~hymnu \emph{Alma Redemptoris Mater}). Nejstarší evropská univerzita v~Boloni užívá motto \emph{Alma mater studiorum}. Odtud dnes spojení \textbf{alma mater} metaforicky označuje univerzitu nebo vysokou školu, na které student získal své vzdělání. \emph{Zdroj: Ottův slovník naučný}}.
    \item SOA záznam si dobře prohlédněte (později budete sami vytvářet SOA záznam pro vlastní doménu).
    \item V~programu Wireshark spusťte zachytávání DNS komunikace znovu (dříve zachycený provoz můžete zahodit).
    \item Pomocí příkazu \texttt{nslookup -type=X vutbr.cz}, kde za \texttt{X} doplňte vhodný typ DNS záznamu, zjistěte autoritativní DNS servery pro doménu \texttt{vutbr.cz}. (Nenechte se zmást tím, že jste dostali neautoritativní odpověď. Diskutujte, abyste si ujasnili pojmy (ne)autoritativní odpověď a autoritativní DNS server pro danou doménu.)
    \item Zastavte zachytávání komunikace v~programu Wireshark.
    \item V~zachyceném provozu nalezněte pakety obsahující komunikaci posledního Vámi provedeného DNS dotazu na doménu \texttt{vutbr.cz} a prozkoumejte je.
    \item Kolik paketů souvisejících s~Vaším dotazem na doménu bylo zachyceno?
    \item Byl proveden rekurzivní nebo iterativní DNS dotaz?
    \item Na jakou IP adresu směřoval paket s~DNS dotazem? Komu náleží tato IP adresa? Pokud netušíte, jakému zařízení IP adresa náleží, zkuste se podívat do souboru \texttt{/etc/resolv.conf}, poté zadejte v~Terminálu příkaz \texttt{ip route} a prohlédněte si výpis a nakonec požádejte cvičícího, ať vám ukáže, jakou IP adresu má na rozhraní \texttt{em0} nastavený učitelský počítač. Diskutujte, jak spolu tyto informace souvisí a proč jste na všech třech místech viděli stejnou IP adresu.
    \item \textbf{Odpovědi sdělte cvičícímu.}
    \item Vyzkoušejte zobrazení DNS záznamů pro další libovolné domény (např. \texttt{MX} záznamy pro \texttt{seznam.cz}~--~všimnněte si priorit u~e-mailových serverů).
\end{enumerate}

\section{Seznámení s~Whois}
\begin{enumerate}
    \item Otevřete ve webovém prohlížeči online nástroj pro prohledávání databáze Whois\\ \url{https://www.nic.cz/whois/}. Btw. Odkud znáte \emph{CZ.NIC}? Co toto sdružení zajišťuje?
    \item Zadejte do vyhledávacího pole doménu \texttt{vutbr.cz}. Prohlédněte si zobrazené informace. Všimněte si například, od jakého roku je doména registrována.
    \item Zjistěte, jaká je veřejná IP adresa Vašeho počítače - například pomocí webového nástroje\\ \url{https://www.whatismyip.com/}. Následně si v~Terminálu pomocí příkazu \texttt{ip addres show} zobrazte IP adresy na rozhraních Vašeho počítače. Proč ani na jednom rozhranní nevidíte svoji veřejnou IP adresu? (nápověda: NAT) Požádejte cvičícího, ať vám ukáže, jakou IP adresu má na rozhraní \texttt{re0} nastaven učitelský počítač. Je to stejná IP adresa jako Vaše veřejná?
    \item Na nové kartě webového prohlížeče otevřete nástroj \url{https://lookup.icann.org/} a svoji veřejnou IP adresu zadejte do vyhledávacího pole. Všimněte si, do jakého rozsahu IP adresa patří a kdo má tento rozsah IP adres přidělen. Btw. Odkud znáte \emph{ICANN}? Co tato organizace zajišťuje?
\end{enumerate}

\section{Blokování vybraných domén}
\begin{enumerate}
	\item Do souboru \texttt{/etc/hosts} přidejte následující řádek:\\
    \verb|0.0.0.0    facebook.com|
    \item Ve webovém prohlížeči zadejte do adresního řádku adresu \url{https://www.facebook.com/}. Podařilo se Vám stránku zobrazit?
    \item Přidejte další záznam do souboru \texttt{/etc/hosts} tak, aby při pokusu o~otevření webové aplikace YouTube došlo k~otevření webové stránky \url{https://www.fit.vutbr.cz/}.
    \item Se spolužáky diskutujte slabiny tohoto řešení omezení přístupu na webové stránky. Jak lze toto řešení (jednoduše) obejít?
\end{enumerate}

\section{Konfigurace vlastního DNS serveru}
\begin{enumerate}
  \item Prostudujte si začátek kapitoly 9.2.3 z~manuálu k~laboratořím.
  \item Váš budoucí DNS server bude spravovat doménu  {\tt isa.cz}.
  \item Nejdříve upravte konfigurační soubor {\tt /etc/named.conf} následujícím způsobem:
    \begin{itemize}
      \item Do parametru {\tt listen-on} v~sekci {\tt options} přidejte svoji IP adresu na rozhraní {\tt enp2s0}
            (tj.~IP adresu 10.10.10.1XX, kde XX je číslo Vašeho počítače).
            Zajistíte tím, že Váš budoucí DNS server bude přijímat a zpracovávat DNS dotazy, které mu na toto rozhraní ({\tt enp2s0}) přijdou. 
      \item Nastavte {\tt allow-query} na {\tt any;}.
            Tím zajistíte, že kdokoliv (v~lokální síti) bude moci poslat na Váš DNS server dotaz a on ho bude zpracovávat.
            Kdybyste nechali v~nastavení možnost {\tt localhost;}, zpracovával by Váš DNS server dotazy pouze z~Vašeho počítače a
            když by mu zaslal DNS dotaz třeba sousední počítač, zahodil by tento dotaz a neodpověděl by.
      \item Téměř na konci souboru vytvořte dvě nové zóny. První zónu pro Vaši doménu {\tt isa.cz} a druhou zónu pro reverzní překlad ({\tt 10.10.10.in-addr.arpa}).
            Podívejte se do laboratorního manuálu, jak má vypadat definice nových zón v~souboru {\tt /etc/named.conf}.
            Cesty k~zónovým souborům nastavíte později -- až po vytvoření těchto souborů.
    \end{itemize}
  
  \item Nyní je potřeba pro nově registrované zóny vytvořit zónové soubory. V~nich používejte vždy FQDN.
  \item Začněte zónovým souborem pro doménu {\tt isa.cz}:
  
    \begin{itemize}
      \item V~souboru {\tt /root/isa3/template.dns.zone} je připravena šablona zónového souboru.
            Zkopírujte tento soubor do složky {\tt /var/named} pod novým jménem {\tt db.isa.cz}.
      \item V~nově vytvořeném souboru {\tt /var/named/db.isa.cz} upravte hodnotu TTL na {\tt 3h}.
      \item Upravte SOA záznam domény {\tt isa.cz} dle manuálu k~laboratořím. Autoritativní server bude {\tt ns1.isa.cz.}
            Email správce bude {\tt admin.isa.cz.} (nelekněte se, že se v~e-mailové adrese místo znaku '{\tt @}' používá znak '{\tt .}').
      \item Vytvořte NS záznam, který bude ukazovat na autoritativní server {\tt ns1.isa.cz}.
      \item Pro autoritativní server {\tt ns1.isa.cz} vytvořte A~záznam, který bude ukazovat na IP adresu Vašeho počítače (na rozhraní {\tt enp2s0}).
            Uvědomte si, že nyní jste pomocí SOA, NS a A~záznamu nastavili, že Váš počítač je tím autoritativním DNS serverem pro doménu {\tt ns1.isa.cz} (tj. Váš počítač spravuje zónový soubor domény).
      \item Přidejte další tři A~záznamy, které budou ukazovat na počítače 01, 02 a 03 v~laboratoři. Záznamy zadejte v~tomto tvaru:
            \verb|PC01    IN    A    10.10.10.101|
      \item Přidejte záznam typu CNAME pro jméno {\tt server} ukazující na {\tt ns1.isa.cz.}
      \item V~případě zájmu nakonfigurujte pro doménu překlad na adresy IPv6 (záznamy AAAA).
    \end{itemize}
    
  \item Tím je zónový soubor pro doménu {\tt isa.cz} připravený. Nyní vytvořte zónový soubor pro reverzní překlad:
  
    \begin{itemize}
      \item V~souboru {\tt /root/isa3/db.127} je připravena šablona zónového souboru pro reverzní překlad.
            Zkopírujte tento soubor do složky {\tt /var/named} pod novým jménem {\tt db.10.10.10.rev}.
      \item V~nově vytvořeném souboru {\tt db.10.10.10.rev} upravte SOA záznam domény. Autoritativní server bude {\tt ns1.isa.cz.}
            Email správce bude {\tt admin.isa.cz.}
      \item Vytvořte NS záznam, který bude ukazovat na autoritativní server {\tt ns1.isa.cz}.
      \item Přidejte další tři PTR záznamy, které budou ukazovat na počítače 01, 02 a 03 v~laboratoři. Záznamy zadejte v~tomto tvaru:
            \verb|101.10.10.10    IN    PTR    PC01|
      \item V~případě zájmu nakonfigurujte pro doménu překlad z~adres IPv6 (záznamy PTR).
    \end{itemize} 
    
  \item Nezapomeňte nyní zaregistrovat zónové soubory v~souboru {\tt /etc/named.conf} (doplnit cesty k~vytvořeným zónám).
  \item Upravte nastavení firewallu, aby propouštěl DNS dotazy. To provedete příkazy \\{\tt iptables -F INPUT}, {\tt iptables -F OUTPUT} a {\tt iptables -F FORWARD}

  \item Spusťte DNS server příkazem {\tt systemctl start named.service}.
    Příkazem {\tt systemctl status named} ověřte, zda byla služba správně spuštěna.
    Případné chyby týkající se chybějící konektivity pomocí protokolu IPv6 ignorujte.

  \item Najděte si někoho do dvojice, s~kým si nyní navzájem vyzkoušíte, že Vaše DNS servery jsou správně nakonfigurované.
        Váš počítač označme jako PCA, počítač Vašeho kolegy jako PCB.
  \item NA PCA i PCB přidejte konfigurační soubor {\tt /etc/dhcp/dhclient-enp2s0.conf} a nastavte
    použití DNS serveru běžícího na PCB, do souboru přidejte řádek \\
    {\tt prepend domain-name-servers 10.10.10.1XX;} \\
    kde XX bude tedy číslo počítače Vašeho kolegy. 
    Na obou počítačích restartujte síťování: {\tt systemctl restart network.service}.
    Ověřte nastavení IP adres a obsah souboru {\tt /etc/resolv.conf}. V~tomto souboru by měla být IP adresa počítače Vašeho kolegy, který se tímto stává Vaším DNS serverem.
  \item V~rogramu Wireshark začněte zachytávat DNS komunikaci na rozhraní {\tt enp2s0}.
  \item V~Terminálu zadejte {\tt ping PC01.isa.cz}. Došlo k~přeložení doménového jména na IP adresu? Pokud ano, gratuluji! Pokud ne, pomozte Vašemu kolegovi najít chybu
    v~konfiguraci jeho DNS serveru.
  \item Prohlédněte si zachycenou DNS komunikaci ve Wiresharku. Nalezněte DNS dotaz a DNS odpověď provedených při {\tt pingu}.
  
  \item {\bf Pochlubte se svými výsledky cvičícímu}.
\end{enumerate}




\section{Ukončení práce v~laboratoři}
\begin{itemize}
  \item Počítač vypněte dávkou {\tt /root/isa3/clean}.
\end{itemize}

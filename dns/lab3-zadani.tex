\section*{Cíle laboratoře}
\begin{itemize}
  \item Seznámení se systémem a protokolem DNS a s~databází Whois.
  \item Konfigurace a spuštění DNS serveru obsluhujícího vlastní doménu.
\end{itemize}

\section*{Základní instrukce}
\begin{itemize}
  \item Přihlaste se do OS CentOS (F3), user/password: {\tt user}/{\tt user4lab}.
  \item Otevřete si příkazovou řádku pro uživatele {\tt user}.
  \item Otevřete si příkazovou řádku pro uživatele {\tt root} příkazem {\tt su}
    (switch user).
  \item Pro editaci konfiguračních souborů použijte libovolný editor (např.
    nano, vim, gedit).
  \item {\bf Odpovědi pište do protokolu.}
\end{itemize}

\section*{Úkoly}

\section{Seznámení s~DNS}
\begin{enumerate}
    \item Spusťte program Wireshark (vždy jako \texttt{root} z~příkazové řádky příkazem \texttt{wireshark \&}) a začněte zachytávat DNS komunikaci na rozhraní, pomocí kterého jste připojeni k~Internetu (\texttt{enp2s0}).
    \item Otevřete terminál a pomocí příkazu \texttt{nslookup -type=a www.vutbr.cz} zjistěte, na jakou IPv4 adresu se překládá doménové jméno \texttt{www.vutbr.cz}. Zjištěnou IPv4 adresu zapište do protokolu ke cvičení.
    \item Zachycený DNS dotaz a odpověď si prohlédněte v~programu Wireshark.
    \item Příkazem \texttt{nslookup -type=soa vutbr.cz} si zobrazte SOA záznam naší Alma mater\footnote{\textbf{Alma mater} (latinsky \textbf{matka živitelka}) je původně antické označení pro bohyni matku. Ve středověké poezii se spojení někdy užívalo i pro Pannu Marii jako \emph{Matku Boží} (např. v~hymnu \emph{Alma Redemptoris Mater}). Nejstarší evropská univerzita v~Boloni užívá motto \emph{Alma mater studiorum}. Odtud dnes spojení \textbf{alma mater} metaforicky označuje univerzitu nebo vysokou školu, na které student získal své vzdělání. \emph{Zdroj: Ottův slovník naučný}}.
    \item SOA záznam si dobře prohlédněte (později budete sami vytvářet SOA záznam pro vlastní doménu).
    \item V~programu Wireshark spusťte zachytávání DNS komunikace znovu (dříve zachycený provoz můžete zahodit).
    \item Pomocí příkazu \texttt{nslookup -type=<X> <doména>} zjistěte autoritativní DNS servery pro doménu \texttt{vutbr.cz}, kde za \texttt{<X>} doplňte vhodný typ DNS záznamu pro zjištění autoritativních serverů a za \texttt{<doména>} doplňte vhodnou doménu. Nenechte se zmást tím, že jste dostali neautoritativní odpověď. Nahlédněte do slidů z~přednášky, abyste si ujasnili pojmy (ne)autoritativní odpověď a autoritativní DNS server pro danou doménu. Své poznatky z odpovědi zapiště do protokolu ke cvičení.
    \item Zastavte zachytávání komunikace v~programu Wireshark.
    \item V~zachyceném provozu nalezněte pakety obsahující komunikaci poslední vámi provedené DNS rezoluce a prozkoumejte je.
    \item Byl proveden rekurzivní nebo iterativní DNS dotaz? Podívejte se do \texttt{Flags} v~DNS dotazu a odpovědi a poznatky zapiště do protokolu ke cvičení.
    \item Na jakou IP adresu směřoval paket s~DNS dotazem? Odpověď zapiště do protokolu ke cvičení.
    \item Zobrazte \texttt{MX} záznamy pro doménu \texttt{fit.vutbr.cz}. Jaký je primární e-mailový server pro doménu \texttt{fit.vutbr.cz}? Správný server zapiště do protokolu ke cvičení.
\end{enumerate}

\section{Seznámení s~Whois}
\begin{enumerate}
    \item Zadejte do Terminálu příkaz \texttt{whois vutbr.cz}. Prohlédněte si zobrazené informace. Od jakého roku je doména registrována? Rok zaznamenejte do protokolu ke cvičení.
    \item Zjistěte, jaká je veřejná IP adresa vašeho počítače - můžete například využít patičku webu \url{www.fit.vut.cz}. Nalezenou adresu zapiště do protokolu ke cvičení. Následně si v~Terminálu pomocí příkazu \texttt{ip address show} zobrazte IP adresy na rozhraních vašeho počítače. IP adresu rozhraní \texttt{enp2s0} zapiště do protokolu ke cvičení. Proč ani na jednom rozhraní nevidíte svoji veřejnou IP adresu? Důvod také zapište do protokolu ke cvičení.
    \item Informace o své veřejné IP adrese zjistěte příkazem \texttt{whois <vase-verejna-IP-adresa>}. Do jakého rozsahu IP adresa patří a kdo má tento rozsah IP adres přidělen? Obě informace zapište do protokolu ke cvičení.
\end{enumerate}

%\section{Blokování vybraných domén}
%\begin{enumerate}
%	\item Do souboru \texttt{/etc/hosts} přidejte následující řádek:\\
%    \verb|0.0.0.0    www.facebook.com|
%    \item Ve webovém prohlížeči zadejte do adresního řádku \url{www.facebook.com}. Podařilo se Vám stránku zobrazit? Pokud ano, restartujte webový prohlížeč, aby se vymazala DNS mezipaměť webového prohlížeče, a zkuste znovu.
%    \item Zamyslete se nad slabinami tohoto řešení omezení přístupu na webové stránky. Jak lze toto řešení (jednoduše) obejít?
%\end{enumerate}

\section{Konfigurace vlastního DNS serveru}
\begin{enumerate}
  \item Prostudujte si začátek kapitoly 9.2.3 z~manuálu k~laboratořím.
  \item Zvolte si vlastní doménu, kterou bude váš budoucí DNS server spravovat. Například {\tt xlogin00.cz.} V zadání bude vaše vybraná doména označována jako {\tt xlogin00.cz}. Vy ale řetězec {\tt xlogin00.cz} v názvech souborů a v konfiguraci nahrazujte svojí vlastní doménou.

\item Nejdříve vytvořte zónový soubor pro doménu {\tt xlogin00.cz} dle následujících instrukcí:
    \begin{itemize}
      \item Soubor {\tt /root/isa3/template.dns.zone} je ukázkový zónový soubor.\\
            Zkopírujte tento soubor do složky {\tt /var/named} pod novým jménem {\tt xlogin00.cz}.
      \item V~nově vytvořeném souboru {\tt /var/named/xlogin00.cz} upravte SOA záznam domény\\ {\tt xlogin00.cz.} Autoritativní server bude {\tt ns1.xlogin00.cz.}
            Email správce bude\\ {\tt admin.xlogin00.cz.} (pozor, v~e-mailové adrese se místo znaku '{\tt @}' používá znak '{\tt .}').
      \item V SOA záznamu aktualizujte sériové číslo, aby odpovídalo dnešnímu datumu ve tvaru {\tt yyyymmdd}.
      \item Vytvořte NS záznam, aby ukazoval na autoritativní server {\tt ns1.xlogin00.cz.}
      \item Pro autoritativní server {\tt ns1.xlogin00.cz.} vytvořte A~záznam, který bude ukazovat na IP adresu Vašeho počítače (na rozhraní {\tt enp2s0}).
            Uvědomte si, že nyní jste pomocí SOA, NS a A~záznamu nastavili, že váš počítač je tím autoritativním DNS serverem pro doménu {\tt xlogin00.cz} (tj. váš počítač spravuje zónový soubor domény).
      \item Přidejte další A~záznam, který bude ukazovat na učitelský počítač v~laboratoři. Záznam zadejte v~tomto tvaru:
            \verb|PCUC    IN    A    10.10.10.1|
      \item Přidejte další A~záznamy, které budou ukazovat na tři libovolné počítače v~laboratoři (např. PC01, PC02 a PC03).
      \item Přidejte záznam typu CNAME pro jméno {\tt server} ukazující na {\tt ns1.xlogin00.cz.}
      \item V~případě zájmu nakonfigurujte pro doménu překlad na adresy IPv6 (záznamy AAAA).
    \end{itemize}
  
  \item Nyní upravte konfigurační soubor {\tt /etc/named.conf}. Při úpravách se můžete inspirovat ukázkovým konfiguračním souborem {\tt /root/isa3/named.conf.local}. Proveďte následující úpravy:
    \begin{itemize}
      \item Vytvořte novou dopřednou zónu (téměř na konci souboru před klíčovým slovem {\tt include}) pro vaši doménu {\tt xlogin00.cz}.
            \item Doplňte absolutní cestu k~zónovému souboru \texttt{xlogin00.cz}
    \end{itemize}
  \item Zkuste spustit DNS server příkazem {\tt systemctl start named.service}.
    Příkazem {\tt systemctl status named} ověřte, zda byla služba správně spuštěna. Chyby související s IPv6 můžete ignorovat.
  
  \item Po úspěšném spuštění DNS serveru nakonfigurujte ještě reverzní překlad pro vaši doménu. Opět začněte vytvoření zónového souboru. 
  

  
  \item Zónový soubor pro reverzní překlad vytvořte dle následujících instrukcí:
    \begin{itemize}
      \item Jako šablonu zónového souboru pro reverzní překlad využijte soubor {\tt /root/isa3/temp\-late.dns.zone}.
            Zkopírujte tento soubor do složky {\tt /var/named} pod názvem \texttt{10.10.10}.
      \item V~nově vytvořeném souboru {\tt /var/named/10.10.10} upravte SOA záznam domény\\ {\tt 10.10.10.in-addr.arpa.} Autoritativní server bude opět {\tt ns1.xlogin00.cz.}
            Email správce bude také opět {\tt admin.xlogin00.cz.}
      \item V SOA záznamu aktualizujte sériové číslo, aby odpovídalo dnešnímu datumu ve tvaru {\tt yyyymmdd}.
      \item Vytvořte NS záznam, který bude ukazovat na autoritativní server {\tt ns1.xlogin00.cz.}
      \item Přidejte PTR záznam, který bude mapovat IP adresu vašeho počítače (na rozhraní {\tt enp2s0}) na doménové jméno autoritativního serveru {\tt ns1.xlogin00.cz.}
      \item Přidejte další PTR záznam, který bude ukazovat na učitelský počítač v~laboratoři. Záznam zadejte v~tomto tvaru:
            \verb|1    IN    PTR    PCUC.xlogin00.cz.|
      \item Přidejte další tři PTR~záznamy, které budou ukazovat na vybrané tři počítače v~laboratoři (např. PC01, PC02 a PC03). Tyto počítače se musí shodovat s počítači vybranými při vytváření dopředného zónového souboru.
      \item V~případě zájmu nakonfigurujte pro doménu překlad z~adres IPv6 (záznamy PTR).
    \end{itemize} 

  \item Vytvořte v~souboru {\tt /etc/named.conf} novou zónu (téměř na konci souboru před klíčovým slovem {\tt include}) pro reverzní překlad ({\tt 10.10.10.in-addr.arpa}) a doplňte absolutní cestu k novému zónovému souboru.
  
  \item Restartujte DNS server příkazem {\tt systemctl restart named.service}.
    Příkazem {\tt systemctl status named} opět ověřte, zda byla služba správně spuštěna. Chyby spjaté s IPv6 můžete i dále ignorovat.

  \item Upravte IP adresu výchozího DNS serveru tímto postupem:
  \begin{itemize}
    \item Na konec souboru {\tt /etc/sysconfig/network-scripts/ifcfg-Wired\char`_connection\char`_1} přidejte následující dva řádky:\\
          \verb|PEERDNS=no|\\
          \verb|DNS1=127.0.0.1|
    \item Vypněte a znovu zapněte síťové rozhraní {\tt enp2s0} příkazy: {\tt ifdown Wired\char`_connection\char`_1}; {\tt ifup Wired\char`_connection\char`_1}
    \item Ověřte, že došlo ke změně výchozího DNS serveru v~souboru {\tt /etc/resolv.conf} a že nyní je výchozím DNS serverem váš počítač.
  \end{itemize}
  \item V~programu Wireshark začněte zachytávat DNS komunikaci na rozhraní {\tt Loopback: lo}.
  \item V~Terminálu zadejte {\tt ping PCUC.xlogin00.cz}. Došlo k~očekávanému přeložení doménového jména na IP adresu? Pokud ano, gratuluji!
  \item Prohlédněte si zachycenou DNS komunikaci ve Wiresharku. Nalezněte DNS dotaz a DNS odpověď provedených při {\tt pingu}.
  \item Otestujte funkčnost dalších DNS záznamů -- například:\\
        {\tt ping server.xlogin00.cz}, {\tt ping PC01.xlogin00.cz}, {\tt nslookup -type=soa xlogin00.cz}, {\tt dig xlogin00.cz}, {\tt dig xlogin00.cz NS}, {\tt nslookup 10.10.10.1}, {\tt nslookup 10.10.10.101}\\
        \\
        Do protokolu ke cvičení zapište výstup příkazu {\tt nslookup 10.10.10.1} a nadepsané části výsledků příkazů {\tt dig PCUC.xlogin00.cz} a {\tt dig -x 10.10.10.1xx}, kde {\tt xx} je číslo vašeho počítače.
  \item Ve dvou instancích Wiresharku začněte zachytávat DNS komunikaci na rozhraních {\tt enp2s0} a {\tt Loopback: lo}. Případně můžete použít jedinou instanci Wiresharku, ale musíte zachytávat na obou rozhraních zároveň. (\texttt{Ctrl + left\_click} k výběru více rozhraní). V takovém případě nezapomeňte zachycený traffic seřadit podle časových razítek.
  
  \item V~Terminálu zadejte {\tt dig www.google.com} (nebo zvolte jiné doménové jméno, které váš vlastní DNS server nezná). Pozor na DNS cache!
  
  \item Zastavte zachytávání DNS provozu v programu Wireshark a analyzujte zachycený provoz. Zjistěte časovou posloupnost dotazů a odpovědí mezi rozhraními, například pomocí časových razítek v sekci \texttt{frame}. Do protokolu ke cvičení tuto informaci zaneste doplněním sekvenčního diagramu. Počet potřebných objektů v diagramu je závislý na zvolené doméně z předchozího kroku a proto ji nad diagram uveďte také. Zamyslete se, kdy byl DNS dotaz iterativní a kdy rekurzivní. 
\end{enumerate}

\section{Ukončení práce v~laboratoři}
\begin{itemize}
  \item Počítač vypněte dávkou {\tt /root/isa3/clean}.
\end{itemize}

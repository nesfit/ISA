
\section*{Cíle laboratoře}
\begin{itemize}
  \item Seznámení se signalizačním protokolem SIP.
  \item Peer-to-peer VoIP pomocí signalizace SIP.
  \item Komunikace VoIP pomocí signalizace SIP přes ústřednu.
\end{itemize}

\section*{Základní instrukce}
\begin{itemize}
  \item Připojte k počítači headset se sluchátky a mikrofonem.
  \item Přihlaste se do OS Kali Linux (F3), user/password {\tt user}/{\tt user4lab}.
  \item V pravé horní části obrazovky {\bf zkontrolujte nastavení zvuku}. Sluchátka a mikrofon nastavte na optimální hlasitost.
  \item Otevřete si příkazovou řádku pro uživatele {\tt user}.
  \item Otevřete si příkazovou řádku pro uživatele {\tt root} příkazem {\tt su}
  (switch user).
  \item V případě potřeby si otevřete další terminál v novém okně.
  \item Pro editaci konfiguračních souborů použijte libovolný editor (např.
  nano, mcedit, vim, gedit).
  \item Pracujte ve dvojicích, zkontrolujte, že máte počítač propojen
    přímým kabelem s přístupovým přepínačem (rozhraní eth0, zdířka E na patch
    panelu).


\end{itemize}

\section*{Úkoly}
\section{Peer-to-peer VoIP pomocí signalizace SIP}
Se svým sousedem nastavte komunikaci VoIP mezi vašimi dvěma počítači bez použití ústředny (spojení peer-to-peer).

\begin{enumerate}
    \item Zapněte program Linphone a vyčkejte až se spustí okno programu.
    \item V menu Options {\bf zrušte} volbu {\bf Always start video}.
    \item V menu Options $\rightarrow$ Preferences $\rightarrow$ Manage SIP accounts nastavte pole Your username na hodnotu {\bf 10XX}, kde {\bf XX je číslo Vašeho počítače}.
    \item Spusťte síťový analyzátor Wireshark a začněte zachytávat pakety na rozhraní, kterým jste připojeni k síti.
    \item Po spuštění nastavte filtr tak, aby zobrazoval pouze protokol SIP (Filter: sip).
    \item V hlavním okně programu zadejte do pole {\bf SIP addres or phone number} SIP adresu Vašeho souseda: {\bf sip:10YY@10.10.10.1YY}, kde {\bf YY} je číslo počítače Vašeho souseda.
    \item Zavolejte Vašemu sousedovi stisknutím zeleného telefonu. Na druhém počítači přijměte hovor, vyzkoušejte, zda se se sousedem navzájem slyšíte a po chvíli hovor ukončete.
(V případě problémů se zvukem zkontrolujte, zda není v systému ztlumen mikrofon, či audio výstup, případně v aplikaci Linphone v menu Options $\rightarrow$ Preferences $\rightarrow$ Multimedia settings zkontrolujte nastavení zvukových zařízení.) 
    \item Proveďte analýzu navazování spojení v odchycených datech ve Wiresharku. Využijte podpory v menu {\bf Telephony $\rightarrow$ VoIP Calls}, kde uvidíte jednotlivé zaznamenané hovory. Vyberte příslušný hovor a pro zobrazení průběhu klikněte na volbu {\bf Flow}.
    \item Zakreslete spojení do grafu v protokolu. Uveďte, pomocí kterých protokolů a mezi jakými IP adresami a porty probíhá signalizace a přenos dat. Zjistěte použitý kodek pro přenos hlasu.
\end{enumerate}
Názvy a čísla podporovaných kodeků lze zobrazit v SIP/SDP zprávě v sekci {\bf Session Initiation Protocol} $\rightarrow$ {\bf Message body} $\rightarrow$ {\bf Session description protocol}:
\begin{figure}[h!]
  \centering
  \fbox{\includegraphics[width=145mm]{img/3a.png}}
\end{figure}

\noindent Informace o tom, který z podporovaných kodeků byl skutečně použit získáte z RTP paketů (Filter: RTP) podle čísla v poli {\bf Payload type}.
\begin{figure}[h!]
  \centering
  \fbox{\includegraphics[width=105mm]{img/3b.png}}
\end{figure}


\section{Komunikace VoIP pomocí signalizace SIP přes ústřednu}
Se svým sousedem nastavte komunikaci VoIP mezi vašimi dvěma počítači pomocí SIP ústředny vašeho ISP.

\subsection{Analýza registrace a odregistrace}
\begin{enumerate}
    \item Spusťte znovu zachytávání paketů v aplikaci Wireshark tlačítkem \includegraphics[width=3mm]{img/ws_start.png}.
    \item Vraťte se do hlavního okna aplikace Linphone. V menu Options $\rightarrow$ Manage SIP accounts zkontrolujte, zda je v poli Your username uloženo {\bf 10XX}, kde {\bf XX je číslo Vašeho počítače}.
    \item Ve stejném okně v sekci {\bf Proxy accounts} klikněte na {\bf Add} a vyplňte následující pole: \\
    ~\\
    {\bf Your SIP identity:} 	{\tt sip:userXX@isa.fit} \\
    {\bf SIP proxy address:}	{\tt sip:10.10.10.222} \\
    ~\\
	Po potvrzení budete vyzvání k zadání uživatelského jména a hesla, zadejte: \\
    {\bf userID:} 	{\tt userXX} \\
    {\bf Password:}	{\tt hesloXX}
    \item Tímto jste provedli registraci k ústředně. Pokud jste vše nastavili správně, mělo by se ve spodní části hlavního okna zobrazit {\bf Registration on ... successfull} \\
    {\it (V případě neúspěchu v nastavení účet odeberte znovu přidejte. Pokud ani toto nepomohlo, ukončene Linphone, spusťte znovu /root/isa4/clean.sh a postupujte znovu od kroku 1)}
    \item Klikněte na tlačítko Zakázat, kterým se odregistrujete ze serveru.
    \item V menu {\bf Options} $\rightarrow$ {\bf Manage SIP accounts} vyberte v sekci {\bf Proxy accounts} vytvořený SIP účet a klikněte na {\bf Remove}, čímž by měla proběhnout odregistrace od ústředny.
    \item Ve Wiresharku analyzujte registraci a odregistraci a zakresleslete jejich průběh do protokolu. Vyplňte požadované údaje a zjistěte, v čem se liší paket, kterým se registrujete, od paketu, kterým se odregistrujete.
\end{enumerate}


\subsection{Analýza hovoru přes ústřednu}
\begin{enumerate}
    \item Znovu se registrujte k ústředně.
    \item Obnovte zachytávání paketů v aplikaci Wireshark tlačítkem  \includegraphics[width=3mm]{img/ws_start.png}.
    \item V hlavním okně programu zadejte do pole {\bf SIP addres or phone number} SIP adresu Vašeho souseda: {\bf sip:10YY@10.10.10.222}, kde YY je opět číslo počítače Vašeho souseda.
    \item Zavolejte Vašemu sousedovi a proveďte analýzu hovoru podobně jako u Peer-to-peer hovoru.
    \item Zakreslete průběh spojení do grafu v protokolu. Uveďte, mezi kterými IP adresami a porty probíhá signalizace a mezi kterými přenos dat. Vyplňte, které protokoly se používají pro signalizaci a které pro přenos. Dále zjistěte, jaký kodek pro přenos hlasu byl použit.
\end{enumerate}

\section{Ukončení práce v laboratoři}
\begin{itemize}
  \item Počítač vypněte jako root skriptem {\tt /root/isa4/clean}.
\end{itemize}

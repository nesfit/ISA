\section{Cíle cvičení}
\begin{itemize}
	\item Seznámení se se základní prací v OS Linux
	\item Seznánemní se se základními tooly pro zjišťování konfigurace zařízení
	\item Analýza síťového provozu pomocí Wireshark
\end{itemize}

\section{Zadání}
Přihlaste se jako uživatel user. Veškeré potřebné příkazy následně spouštějte jako \texttt{sudo}.
\subsection{Zjišťování konfigurace}
V této části se budeme zabývat převážně síťovou konfigurací systému.
\begin{enumerate}
\item Vypište konfiguraci vašeho stroje (IP adresu, masku, síť, broadcastovou adresu, default gateway).
\item Otestujte konektivitu k default gateway, následně konektivitu do internetu.
\item Zobrazte si záznamy v routovací a arp tabulce.
\item Vypište implicitní servery DNS.
\item Upravte patřičný soubor tak, aby po spuštění příkazu \texttt{ping google}, byl ping proveden vůči ip adrese 8.8.8.8. Zapište jak a který soubor jste upravili.
\item Vypište aktivní tcp spojení, vyberte jeden záznam, zapište si ho a popište význam jednotlivých položek.
\item Zobrazte události zaznamenané v logu za dnešní den.
\end{enumerate}

\subsection{Wireshark}
V této části cvičení se budeme zabývat analýzou a zachytáváním provozu v programu Wireshark. Spuštění wireshark provedete příkazem \texttt{wireshark-gtk} jako \texttt{sudo}.
\begin{enumerate}
\item Pomocí programu wireshark začněte zachytávat pouze HTTP komunikaci. (Jakmile začnete zachytávat, spusťe si prohlížeč a načtěte stránku \texttt{http://www.fit.vutbr.cz}. Zachycený provoz uložte do svého domovského adresáře.
\item Zahajte zachytávání komunikace. Odstraňte ARP záznamy (příkaz \texttt{arp}). Zobrazte veškerou komunikaci, následně vyfiltrujte pouze ARP a ICMP pakety. Vygenerujte ICMP komunikaci. Analyzujte obsah ARP paketů, zapište obsah sender, target mac a ip adresy pro dvojici ARP request a response. Zapište co jste zadali do filtru.
\item Ve wireshark si otevřte soubor s příponou \texttt{.pcap} z adresáře \texttt{isa1}. Zobrazte si graf síťových toků a zakreslete jej.
\item Zachyťte pouze HTTP a DNS provoz. Ve webovém prohlížeči zkuste otevřít několik stránek na různých URL adresách. Analyzujte obsah a posloupnost DNS paketů a následných HTTP paketů. Zkuste opakovaně načítat stejnou stránku. Proč v některých případech není zachycena DNS komuniace?
\end{enumerate}

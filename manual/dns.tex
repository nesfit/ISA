%%%%%%%%%%%%%%%%%%%%%%%%%%%%%%%%%%%%%%
\section{Domain Name System}
\label{dns}
Cílem DNS je zajistit překlad mezi doménovým jménem a IP adresou. Dříve se jednalo především o~překlad hostname na IP adresu a naopak, tedy záznamy typu {\tt A}, {\tt AAAA}, {\tt PTR} \cite{rfc1035,rfc3596}. Dalším známým typem je {\tt MX}, který deleguje zodpovědnost za příjem e-mailové pošty dané domény. Tento typ se od předchozích tří odlišuje, protože se nejedná o~adresaci hosta, ale služby. Podobný význam mají i záznamy typu {\tt SRV}, které lze v~současnosti využít pro služby SIP a XMPP \cite{rfc2782}.

Mimo tyto typy záznamů existují i další. Některé jsou definovány již v~původním návrhu, jiné přidané později nebo význam původních typů je využit pro další služby (např. {\tt TXT} se používá pro distribuci veřejného klíče podepisování emailu DKIM \cite{rfc4871}).

\subsection{Klient}
Aby klient věděl, na který server se obrátit s~požadavkem na přeložení doménového jména na IP adresu či opačně, je součástí systému tzv. resolver. Tento resolver má několik konfiguračních souborů, z~nichž jmenujme {\tt /etc/hosts}, {\tt /etc/resolv.conf} a {\tt /etc/host.conf}.

První z~těchto souborů se používá pro statický překlad doménového jména na IP adresu. Formát souboru a další popis lze nalézt v~manuálových stránkách {\tt man hosts}.

Dynamický překlad, neboli překlad pomocí DNS serveru, vyžaduje existenci konfiguračního souboru {\tt /etc/resolv.conf}, který obvykle obsahuje jedenkrát volbu {\tt search} a jednu nebo více voleb {\tt nameserver}.
\begin{description}
  \item[search] definuje tzv. searchlist, neboli seznam domén (max. 6), které se budou prohledávat, pokud je požadavek na překlad neúplného doménového jména. Tedy má-li tato volba podobu {\tt search fit.vutbr.cz}, pak při pokusu přeložit jméno {\em merlin} se při neúspěchu pokusí systém také přeložit jméno {\em merlin.fit.vutbr.cz}.
  \item[nameserver] se uvádí právě s~jedním parametrem, který definuje IP adresu DNS serveru, který se systém pokusí kontaktovat. Může se jednat jak o~IPv4 tak o~IPv6 adresu.
\end{description}
Další možnosti, které může soubor obsahovat, lze nalézt v~manuálových stránkách {\tt man resolv.conf}.

V~souboru {\tt /etc/host.conf} lze definovat, v~jakém pořadí se předchozí dvě volby využijí. Standardně se nejprve zkouší statický překlad a poté dynamický. Toto výchozí nastavení umožňuje lokální předefinování překladu z doménového jména na IP adresu.

Pokud se síť na klientovi konfiguruje dynamicky, pak je soubor {\tt /etc/resolv.conf} upravován automaticky. Při statické konfiguraci musí být obsah souboru upraven ručně.

\subsection{Konfigurace serveru}
Konfigurace DNS se stává ze dvou částí -- konfigurace vlastností samotné aplikace a konfigurace obsluhovaných zón. Existuje mnoho různých variant implementací. V~laboratoři se bude pracovat s~implementací od ISC -- BIND.

Konfigurační soubor {\tt\bf named.conf} se na počítačích v~laboratořích nachází ve složce {\tt
/etc/}.

%\begin{verbatim}
%
%acl "local" {
%  <prefix>/<délka prefixu>;
%};
%
%options {
%  listen-on-v6 { <ipv6 adresa lokálního rozhraní>; };
%  allow-query { local; };
%  allow-recursion { local; };
%  allow-transfer { none; };
%  allow-update { none; };
%  directory "/etc/namedb/working";
%  pid-file "/var/run/named/pid";
%}
%
%zone "." IN {
%  type hint;
%  file "/etc/namedb/named.root";
%};
%
%zone "localhost" IN {
%  type master;
%  file "/etc/namedb/master/localhost-forward.db";
%  notify no;
%};
%
%zone "127.in-addr.arpa" IN {
%  type master;
%  file "/etc/namedb/master/localhost-reverse.db";
%  notify no;
%};
%
%zone "0.ip6.arpa" IN {
%  type master;
%  file "/etc/namedb/master/localhost-reverse.db";
%  notify no;
%};
%
%zone "255.in-addr.arpa" IN {
%  type master;
%  file "/etc/namedb/master/empty.db";
%  notify no;
%};
%
%zone "0.in-addr.arpa" IN {
%  type master;
%  file "/etc/namedb/master/empty.db";
%  notify no;
%};
%
%\end{verbatim}

%Zóna typu {\tt hint} odkazuje na seznam tzv. root serverů, tedy takových serverů, které znají informace
%o všech registrovaných top-level doménách a dokáží říci, které servery jsou za tyto top-level domény
%zodpovědné. Zóna typu {\tt master} značí, že server je autoritativním serverem pro danou doménu.
%Mezi povinné záznamy pro takovou doménu patří záznam typu SOA a alespoň jeden NS odkazující na server samotný. Zároveň by měl být obsažen i záznam typu A pro tento server.
Další možnosti konfiguračního souboru naleznete v~manuálových stránkách {\tt man named.conf}.

\subsubsection{Zóna typu hint}
Asociuje se s~nejvyšší doménou v~rámci celé hierarchie. Pokud přijde na server požadavek, prohledá seznam spravovaných zón ostatních typů ({\em master}, {\em slave}, {\em forward}). Pokud záznam nenajde, vybere jeden z~kořenových serverů definovaný právě v~tomto typu zóny.

Soubor, který je vyžadován pro konfiguraci této domény lze získat například programem dig:
\begin{verbatim}
 dig +norec NS . @a.root-servers.net > /var/named/db.root
\end{verbatim}

\subsubsection{Lokálně obsluhovane zóny}
Tyto zóny lze rozdělit do dvou skupin -- {\bf loopback adresy} a {\bf ,,prázdné zóny''}. RFC 6303
\cite{rfc6303} je z~kategorie {\em best practice} a definuje, které zóny by měl být schopen DNS server
obsloužit sám a tím redukovat dotazy přeposílané na další servery. Ve většině případů jsou to zóny pro reverzní záznamy privátních IP adres.

%Výchozí soubor aplikace {\bf named} ve FreeBSD respektuje právě RFC 6303 a navíc doplňuje ještě další
%zóny. Například nealokované rozsahy IPv6 adres. Pro práci v~laboratoři nahraďte původní soubor {\tt
%/etc/namedb/named.conf} souborem {\tt /root/isa1/named.conf}, který je zkrácen a obsahuje pouze nezbytné definice.

\subsubsection{Vlastní zóna}\label{sec:vlastni_zona} 
Aby server začal překládat vlastní zónu, je třeba přidat definici zóny do souboru {\tt /etc/named.conf} a vytvořit zónový soubor. Pokud se přidává zónový soubor pro doménu, je vhodné přidat i zónu pro reverzní záznamy.

\begin{verbatim}
zone "moje.domena.cz" {
  type master;
  file "/var/named/moje.domena.cz.zone";
};

zone "0.168.192.in-addr.arpa" {
  type master;
  file "/var/named/192.168.0.rev";
};
\end{verbatim}

Pro každou z~těchto zón se tvoří samostatný zónový soubor, jehož základní tvar má podobu:

%$ORIGIN <nazev domény>
\begin{verbatim}
$TTL 5m

@ IN  SOA <fqdn autoritativniho serveru>. <email správce>. (
  1   ; seriové číslo
  10h ; obnovovací interval pro sekundární servery
  10m ; prodleva, po které se sekundární server pokusí znova kontaktovat
      ; autoritativní server, pokud se předchozi spojení nezdařilo
  1w  ; jestliže se sekundárnímu serveru nepodaří kontaktovat autoritativní
      ; server během této doby, přestane vracet záznamy pro tuto doménu
  1h  ; původně výchozí TTL (nahrazeno $TTL), nově určuje
      ; NEGATIVNÍ TTL -- pro chybové odpovědi (např. NXDOMAIN)
  )

@ IN  NS <fqdn autoritativního serveru>
@ IN  NS <fqdn sekundárního serveru>

následuje seznam dalších záznamů
\end{verbatim}
Typ záznamu {\tt\bf SOA} definuje parametry zóny -- jméno autoritativního serveru a email správce domény. Všimněte si, že znak `{\tt @}' má zvláštní význam (zastupuje název domény). Z~tohoto důvodu se v~emailu správce nahrazuje znak `{\tt @}' za znak `{\tt .}' (tečka).

Záznam typu {\tt\bf NS} by měl být vždy obsažen alespoň jednou a to právě pro autoritativní server. Počet sekundárních serverů je libovolný. Tyto záznamy by také měly byt obsaženy v~nadřazené doméně, aby bylo možné nalézt server zodpovědný za danou poddoménu. Tedy kořenová doména obsahuje {\tt NS} záznamy pro domény nejvyšší úrovně {\em .cz, .com, .eu, \dots} a ty zase pro jednotlivé poddomény {\em google.com, seznam.cz, \dots}.

V~zónovém souboru se rozlišuje mezi relativním a absolutním jménem. O~jaký typ jména jde, se rozlišuje podle zakončení. Končí-li tečkou, je to jméno absolutní (plně kvalifikované). V~opačném případě jde o~jméno relativní a za to se vždy doplní obsah definován v~{\tt \$ORIGIN}. Jestliže není definován, použije se název, který je definován u~názvu zóny v~souboru {\tt named.conf}.

Další záznamy mají obecný tvar:
\begin{verbatim}
jméno  ttl  třída  typ  <na typu závislá data>
\end{verbatim}
Jméno běžně bývá ve tvaru relativním. Pokud není hodnota TTL jiná než je výchozí pro celou zónu, pak se může vynechat. Třída záznamu se běžně používá pouze {\tt IN} (Internet). Kromě typu SOA a NS, se v~laboratorním cvičení použijí záznamy typu {\tt A}, {\tt PTR} a volitelně {\tt AAAA}. Typ {\tt AAAA} se chová stejně jako záznam typu {\tt A}, který se používá pro IPv4. Typově závislými daty je v~tomto případě adresa IPv6, resp. IPv4. Příklad:
\begin{verbatim}
host  IN A  192.168.0.1
\end{verbatim}
\begin{verbatim}
host  IN AAAA  fd12:1234::1
\end{verbatim}
Typ {\tt PTR} nerozlišuje, zda se jedná o~IPv6 nebo IPv4 adresu. Pro tento případ se typově závislá data chovají stejně jako jméno na začátku záznamu. V~tomto případě se ale vždy zapisuje v~plném tvaru. Další zvláštností PTR záznamu, který je vidět i v~názvu zóny, je, že se hodnoty zapisují v~obráceném pořadí a každá hodnota je oddělená tečkou. Obrácené pořadí je kvůli vyhodnocování záznamů od konce a tečka mezi každým číslem dovoluje snadnější delegaci podsítě na jiný DNS server. Příklad:
\begin{verbatim}
1		IN	PTR	host.moje.domena.cz.
\end{verbatim}

Pro IPv6 můžete předefinovat proměnnou {\tt \$ORIGIN} např. následovně:
\begin{verbatim}
$ORIGIN 0.0.0.0.0.0.0.0.4.3.2.1.2.1.d.f.ip6.arpa.
1.0.0.0.0.0.0.0.0.0.0.0.0.0.0.0   IN PTR host.moje.domena.cz.
\end{verbatim}
Pozor však na to, aby expanzí nedocházelo k vytváření neplatných záznamů jako jsou
\begin{verbatim}
$ORIGIN 0.0.0.0.0.0.0.0.4.3.2.1.2.1.d.f.ip6.arpa.
2.0.0.0.0.0.0.0.0.0.0.0.0.0.0.0   IN PTR host
1.0.0.0.0.0.0.0.0.0.0.0.0.0.0.0   IN PTR host.moje.domena.cz
\end{verbatim}
protože po expanzi se z nich stanou následující záznamy:
{\small
\begin{verbatim}
2.0.0.0.0.0.0.0.0.0.0.0.0.0.0.0.0.0.0.0.0.0.0.0.4.3.2.1.2.1.d.f.ip6.arpa. IN PTR
                                              host.0.0.0.0.0.0.0.0.4.3.2.1.2.1.d.f.ip6.arpa.
1.0.0.0.0.0.0.0.0.0.0.0.0.0.0.0.0.0.0.0.0.0.0.0.4.3.2.1.2.1.d.f.ip6.arpa. IN PTR
                               host.moje.domena.cz.0.0.0.0.0.0.0.0.4.3.2.1.2.1.d.f.ip6.arpa.
\end{verbatim}}

Příklad PTR záznamu pro IPv4 je:
\begin{verbatim}
19.176.229.147.in-addr.arpa. IN	PTR	merlin.fit.vutbr.cz.
\end{verbatim}

Po každé změně zónového souboru by se vždy mělo upravit i sériové číslo zóny a to tak, aby bylo vždy větší než předchozí hodnota. Také je nutné aplikaci načíst změněnou konfiguraci.
\begin{verbatim}
systemctl restart named.service
\end{verbatim}

\subsection{DNS Over TLS}\label{DoT}
DNS Over TLS (DOT) přináší šifrování a autentizaci do DNS komunikace. Tento protokol je definován v RFC 7858\cite{RFC7858}. Protokol má přidělen standardizovaný port 853 jako defaultní. TLS protokol zde zaobaluje DNS požadavky a odpovědi jako přenášený payload viz obrázek \ref{fig:dot-encapsulation}.

\begin{figure}[h]
    \centering
    \includegraphics[scale=1]{fig/dot.pdf}
    \caption{DNS Over TLS zapouzdření.}
    \label{fig:dot-encapsulation}
\end{figure}

Komunikace probíhá pomocí klasického klient-server schématu a transportním protokolem je TCP. Nejprve proběhne TCP handshake, následuje TLS handshake tak jak je definován v RFC 5246\cite{RFC5246}. Jakmile je spojení ustaveno dojde k přenosu dat.

Pro minimalizaci latence klienti nemusí čekat na odpovědi serveru, ale můžou použít pipelining pro odesílání více dotazů v rámci jednoho TLS spojení. Přenášená data uvnitř TLS jsou stejného formátu jako v případě standardního DNS přenášeného přes TCP transportní protokol.\cite{RFC7858}

Resolvíng pomocí DNS over TLS lze na Linuxových systémech nastavit například v podobě spuštění Unbound DNS Caching serveru. Kde je nutné upravit konfigurační soubor \texttt{/etc/unbound/unbound.conf} tak, aby obsahoval následující:

\begin{verbatim}
forward-zone:
    name: "."
    forward-ssl-upstream: yes
    # Cloudflare DNS
    forward-addr: 1.1.1.1@853
    forward-addr: 1.0.0.1@853
\end{verbatim}

Následně službu zapnout a nastavit systémový resolver na adresu \texttt{127.0.0.1}.

\subsection{DNS Over HTTPS}\label{DoH}
DNS Over HTTPS (DOH) je nový standard pro bezpečný a šifrovanz přenos DNS provozu tunelovaný v protokolu HTTPS. Tento protokol nemá vlastní definovaný port a je přenášen jako ostatní HTTPS provoz pod portem 443. DOH je definován standardem vydaným roku 2018 skrývajícím se pod dokumentem označeným jako RFC 8484 \cite{RFC8484}.

\begin{figure}[h]
    \centering
    \includegraphics[scale=0.75]{fig/doh-encapsulation.pdf}
    \caption{DNS over HTTPS zapouzdření.}
    \label{fig:doh-encapsulation}
\end{figure}

DNS zprávy jsou zde přenášeny standardními metodami HTTP GET a POST. V případě GET je DNS request zakódován ve formátu URL64Base jako parametr URI. Výsledná odpověď je obdržena v těle HTTP odpovědi ve standardním DNS wire formátu. V případě metody POST je již sám dotaz zakódován v DNS wire formátu a přenášen v těle POST dotazu.


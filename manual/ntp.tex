%%%%%%%%%%%%%%%%%%%%%%%%%%%%%%%%%%%%%%
\section{Synchronizace času protokolem NTP (Network Time Protocol)}\label{ntp}
Protokol NTP \cite{rfc1305} umožňuje synchronizovat čas mezi uzly v~síti. Jeho úkolem je přenášet časové informace z primárních časových serverů s přesnými hodinami na ostatní servery a klienty. Protokol NTP při synchronizaci bere v úvahu proměnlivou dobou přenosu paketu, zpoždění paketu a další parametry. NTP organizuje servery hierarchicky do úrovní: primární servery NTP předávají časové informace sekundárním serverům, ty je pak předávají serverům nižší úrovně atd. Úroveň vzdálenosti od přesného zdroje času se vyjadřuje veličinou {\em stratum}. Nejnižší hodnota 0 označuje referenční zdroj přesného času (např. GPS). Stratum 1 označuje primární servery, které jsou synchronizovány právě s~referenčním zdrojem. Stratum 2 jsou servery synchronizovány se servery stratum 1 atd. \cite{rfc5905}.

Implementaci protokolu NTP zajišťuje balík aplikací {\em ntp}. Tento balík se skládá z~několika
aplikací. V~rámci laboratorního cvičení budeme používat aplikace {\tt ntpd} či {\tt ntpq} (viz část \ref{sec:ntpq}). Chrony\footnote{Viz \url{https://chrony.tuxfamily.org/}.} nabízí alternativní
implementaci NTP.

\subsection{Aplikace {\tt ntpd}}
Aplikace {\tt ntpd} (NTP démon) slouží k synchronizaci času pomocí NTP. Může pracovat v režimu klienta i serveru. Aplikace běží na pozadí a neustále synchronizuje čas s~nastavenými servery, případně upravuje lokální čas. Úprava lokálního času se provádí úpravou rychlosti běhu lokálního času. Pokud je lokální čas pozadu, resp. se předbíhá, tak se systémové hodiny {\em zrychlují}, resp. {\em zpomalují}. Tento způsob úpravy času znamená, že pokud je čas odchýlen o~několik minut, bude nějakou dobu trvat, než dojde k synchronizaci. Na druhou stranu se tak zabrání skokové změně a navíc čas se nikdy neposune do minulosti. Pokud je lokální čas odchýlen o~více než 1000 sekund (necelých 17 minut), aplikace to vyhodnotí jako chybu a skončí. Pokud se tak stane, objeví se zpráva v~systémovém logu.

Zda aplikace běží, lze zjistit například příkazem {\tt ntpq -p}. Pokud není démon {\tt ntpd} spuštěn, skončí aplikace hláškou {\em ntpq: read: Connection refused}. 

Službu NTP je možné spustit příkazem:
\begin{verbatim}
systemctl start ntp.service
\end{verbatim}
Aby se předešlo problému v~případě, kdy např. hardwarové hodiny jdou špatně a při vypnutí může dojít k~odchýlení lokálního času o~více než 17 minut, je možné vynutit okamžité nastavení času příkazem
\begin{verbatim}
ntpd -qg  # -g vypne kontrolu odchylky a nastaví čas bez ohledu na velikost odchylky
          # -q ukončí běh démona poté, co byl čas nastaven
\end{verbatim}

\subsubsection{Konfigurace}
Základním konfiguračním souborem je {\tt /etc/ntp.conf}. Tento konfigurační soubor obsahuje mnoho
konfiguračních voleb. Nastavení serverů NTP zajišťuje volba {\tt server}, za níž následuje adresa nebo jméno NTP serveru. Tato volba se může vyskytovat v konfiguračním souboru opakovaně. Pomocí této volby se aplikace {\tt ntpd} přepne do role klienta, který může využít k sychronizaci svého času více serverů NTP. V režimu klient se lokální hodiny synchronizují podle času ze vzdáleného serveru NTP. 

Kromě režimu klient může {\tt ntpd} pracovat v režimu {\em peer}, která umožňuje, aby se vzdálený server synchronizoval podle lokálních hodin. To je užitečné v případě, že je v síti více serverů NTP, které mohou mít lepší zdroj času. Zároveň to řeší problém výpadků serverů. 

V konfiguraci můžeme také zvolit možnosti komunikace typu {\em broadcast} nebo {\em manycastclient}, za nimiž následuje broadcastová či multicastová adresa. Na tyto adresy server NTP posílá periodické zprávy s aktuálním stavem hodin. Pro multicastové přenosy se využívají skupiny definované IANA pro protokol NTP, tj. 224.0.1.1 pro IPv4 a ff05::101 pro IPv6. 

Jinou užitečnou volbou je {\em restrict}, která slouží pro řízení přístupu. Aplikací {\tt ntpd} mohou být zaslány různé požadavky přes síť (například pomocí {\tt ntpq}). Je vhodné povolit některé dotazy jen z~určitého uzlu nebo podsítě. Využití této volby může být také užitečné v~případě, že se pro nastavování času využívá broadcastu nebo multicastu a není žádoucí, aby takto vyslanou informaci klienti akceptovali z~libovolného zdroje. Základní tvar tohoto příkazu je:
\begin{verbatim}
restrict <adresa> [mask <maska>] [<jeden či více příznaků>]
\end{verbatim}

Příznak definuje omezení pro danou adresu/síť:
\begin{itemize}
  \item ignore -- zahazovat všechny pakety
  \item nomodify -- povolí pouze dotazy, požadavky měnící stav serveru jsou zahazovány
  \item noquery -- zakáže dotazy pomocí {\tt ntpq}, synchronizace času není ovlivněna
  \item notrust -- zahazovat neautentizované pakety
\end{itemize}
Další příznaky a jejich popis lze nalézt v~manuálových stránkách {\tt man ntp.conf}.

%%Pokud budou aplikace {\tt ntpq} a {\tt ntpdc} (viz podkapitola \ref{sec:ntpq} používané i pro změnu konfigurace, pak je nezbytné nastavit autentizaci. Protokol nabízí možnost využití symetrické i asymetrické kryptografie. Pro použití symetrické kryptografie jsou k~dispozici tyto volby:
%%\begin{itemize}
%%  \item keys -- tato volba má jeden parametr, který udává název souboru, který obsahuje používané klíče (obvykle {\tt /etc/ntp.keys},
%%  \item trustedkey -- výčet klíčů, kterým se bude důvěřovat,
%%  \item requestkey -- seznam klíčů, které mohou být použity aplikací {\tt ntpdc},
%%  \item controlkey -- seznam klíčů, které mohou být použity aplikací {\tt ntpq}.
%%\end{itemize}
%%Formát souboru {\tt /etc/ntp.keys} má následující tvar:
%%\begin{verbatim}
%%<číslo klíče> <typ> <heslo>
%%\end{verbatim}
%%Typ může nabývat čtyř hodnot. Hodnoty {\tt S} a {\tt N} používají bitový formát a běžně se neužívají. Hodnoty {\tt A} a {\tt M} používají textový řetězce délky 1 až 8 znaků a určují způsob zašifrování při přenosu. Nejčastěji se užívá možnost {\tt M}, která značí použití DES nebo MD5. Příklad souboru pak může vypadat následovně:
%%\begin{verbatim}
%%1 M heslo1
%%2 M secret
%%3 M passwd
%%\end{verbatim}
%%Konfigurace v~{\tt /etc/ntp.conf} potom vypadá:
%%\begin{verbatim}
%%keys /etc/ntp.keys
%%trustedkey 1 2 3
%%requestkey 2
%%controlkey 1 3
%%\end{verbatim}
%%
%%Toto značí, že důvěryhodné jsou všechny tři klíče. Aplikace {\tt ntpdc} se může autentizovat pouzde klíčem 3 a aplikace {\tt ntpq} klíči 1 a 3.   
%%
  
\subsection{Aplikace {\tt ntpq}}\label{sec:ntpq}
%% Tyto dvě aplikace nabízejí v~podstatě podobné možnosti konfigurace serveru NTP. Rozdílů je mezi těmito programy několik. První, který byl již uveden výše, je v~tom, že každá z~těchto aplikací může používat jinou množinu klíčů. Podstatnější rozdíl z~uživatelského hlediska je v~tom, že aplikace {\tt ntpdc} má přesně definovaný výčet příkazů, které lze aplikovat, a proto může měnit jen to, co je v~aplikaci definováno. Naproti tomu {\tt ntpq} disponuje příkazem {\tt :config}, kterému se jako parametry předávají konfigurační volby, které se používají v~souboru {\tt /etc/ntp.conf}. Další rozdíl je ve formátu zpráv, pomocí kterých komunikují se serverem.

Aplikace {\tt ntpq} (NTP query) se používá pro zasílání dotazů serveru NTP {\tt ntpd} z důvodu sledování běhu a výkonnosti, případně zjišťování aktuálního stavu. {\tt ntpq} zasílá zprávy přes síťové rozhraní, i když běží lokálně či vzdáleně. Tato aplikace může měnit konfiguraci {\tt ntpd} za běhu. Možnost změny parametru přes síť může být nežádoucí, proto je vhodné omezit pomocí volby {\em restrict} přístup pouze přes rozhraní {\tt loopback}. 

Příklad výstupu volání {\tt ntpq -p}:

\begin{verbatim}
     remote           refid      st t when poll reach   delay   offset  jitter
==============================================================================
+rhino.cis.vutbr 248.205.243.78   3 u  425 1024  377   10.070   -4.280  10.575
+tik.cesnet.cz   195.113.144.238  2 u  622 1024  377    4.796   -3.464  16.528
*tak.cesnet.cz   .GPS.            1 u  364 1024  377    5.008   -3.625   6.228
\end{verbatim}

Výpis obsahuje následující hodnoty:

\begin{description}

  \item[remote] Klient se synchronizuje vůči třem serverům: rhino.cis.vutbr.cz,
    tik.cesnet.cz a tak.cesnet.cz. Hvězdičkou je označený primární zdroj času,
    plusem sekundární zdroje času.

  \item[refid] Primární zdroj času pro vzdálený server NTP.

  \item[stratum] Počet skoků od vzdáleného serveru k přesnému zdroji času (1
    znamená, že zdroj přesného času je přímo připojen, 16 znamená, že vzdálený
    server je nedosažitelný).

  \item[when] Počet sekund od posledního kontaktu se serverem.

  \item[poll] Počet sekund mezi jednotlivými dotazy protokolem NTP.

  \item[reachability] Úspěšnost posledním 8 pokusů o kontakt vzdáleného serveru
    v osmičkové soustavě. (1 znamená pouze poslední pokus uspěl, 377 znamená
    všech 8 posledních pokusů uspělo).

  \item[delay, offset, jitter] Zpoždění, posun a jitter -- charakteristiky
    vzdáleného zdroje času, podle kterého se volí primární a sekundární zdroje
    času.
\end{description}

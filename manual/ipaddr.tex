\section{Adresy v~IPv4 sítí}\label{adresy_ipv4}

\subsection{Formát adresy}
IPv4 adresa je délky 32 bitů a preferovaný zápis má formát {\tt X.X.X.X}, kde
{\tt X} je decimální zápis 8 bitového čísla. Příklad:\\
\begin{verbatim}
8.8.8.8
127.0.0.1
\end{verbatim}

Adresa dvě části:
\begin{itemize}
    \item adresa sítě ({\bf prefix}) a
    \item adresa uzlu.
\end{itemize}
Délka prefixu se zapisuje v~desítkovém tvaru za lomítko. {\tt 192.168.0.1/24}

V~síti s~daným prefixem existují 2 speciální adresy, které nelze použít pro
adresování jednotlivých uzlů:
\begin{itemize}
    \item adresa sítě - adresa uzlu je nulová, např. 192.168.0.0/24
    \item broadcastová adresa - nejvyšší možná adresa uzlu pro daný prefix,
        např. 192.168.0.255/24
\end{itemize}

\section{Adresy v~IPv6 sítí}\label{adresy_ipv6}

\subsection{Formát adresy}
IPv6 adresa je délky 128 bitů a zapsaných ve formátu {\tt X:X:X:X:X:X:X:X}, kde
{\tt X} je hexadecimální zápis 16 bitového čísla.
Příklad:\\
\begin{verbatim}
FEDC:BA98:7654:3210:fedc:ba98:7654:3210
1080:0000:0000:0000:0008:0800:200C:417a
\end{verbatim}

Preferovaný formát je dále upraven v~RFC 5952\footnote{https://tools.ietf.org/html/rfc5952}, které definuje následující
pravidla:
\begin{itemize}
    \item Nuly na začátku každého 16 bitového čísla je potřeba vynechat (např.
        80 namísto 0080 a 0 namísto 0000).
    \item Více nulových bloků lze nejvýše jednou nahradit znakem {\tt ::}, a
        MUSÍ být nahrazena nejdelší možná posloupnost takových nulových bloků.
        V~případě více shodně dlouhých posloupností, nahrazuje se ta nejvíce
        vlevo.
    \item Znak {\tt ::} nesmí nahrazovat samostatný nulový blok.
    \item Znaky "a", "b", "c", "d", "e", "f" hexadecimální soustavy se vždy
        píšou malými písmeny.
\end{itemize}
Adresy z~předchozího příkladu tedy budou vypadat následovně:\\
\begin{verbatim}
fedc:ba98:7654:3210:fedc:ba98:7654:3210
1080::8:800:200c:417a
\end{verbatim}

Stejně jako v~IPv4 má adresa dvě části: \emph{adresa sítě ({\bf prefix})} a
\emph{adresa uzlu}. Délká prefixu se zapisuje v~desítkovém tvaru za lomítko
(např. {\tt 1080::/60}).

\subsection{Rozdělení adres}
IPv6 adresy je možno rozdělit podle rozsahu. Typicky může jít o~tři
možnosti -- adresy na lince (neprojdou za router), lokální adresy (ULA, nejsou
routovatelné ve veřejné sítí) a veřejné adresy.

\begin{table}[ht!]
    \begin{center}
        \begin{tabular}{l|l|l}
            Význam & Prefix (bitově) & Prefix \\
            \hline\hline
            Veřejné  & {\tt 001} & {\tt 2000::/3} \\
            \hline
            Lokální  & {\tt 1111 110} & {\tt FC00::/7} \\
            \hline
            Linkové  & {\tt 1111 1110 10} & {\tt FE80::/10} \\
            \hline
            Multicast & {\tt 1111 1111} & {\tt FF00::/8} \\
            \hline
        \end{tabular}
    \end{center}
\end{table}

\subsection{Lokální IPv6 (ULA) adresy}\label{ula}
Síťová část ULA adresy se stává ze 4 části:
\begin{description}
    \item [Prefix] {\tt fc00::/7}
    \item [L] bit 1 pokud byl prefix přiřazen lokálně, 0 zatím nebyla
        definována, začátek adresy je proto typicky {\tt fd00::/8}
    \item [Global ID] identifikátor sítě, měl by být unikátní, standard
        popisuje pseudonáhodný algoritmus pro generování. Unikátnost je
        požadována, aby při spojení více lokálních sítí nebylo třeba žádnou
        přečíslovat.
    \item [Inderface ID] Identifikátor rozhraní, existuje několik variant jak
        jej získat. Původní standard (RFC
        3513\footnote{https://tools.ietf.org/html/rfc3513} a RFC
        4291\footnote{https://tools.ietf.org/html/rfc4291}) doporučoval použití
        EUI-64. V~současnosti byla tato metoda nahrazena RFC
        7217\footnote{https://tools.ietf.org/html/rfc7217}. Pro ochranu
        soukromí jsou další specifika automatické generace identifikátorů
        definovány v~RFC 4941\footnote{https://tools.ietf.org/html/rfc4941}.
\end{description}

\begin{table}[ht!]
    \begin{center}
        \begin{tabular}{c|c|c|c|c}
            7 bits & 1 &  40 bits  &  16 bits  & 64 bits \\
            \hline
            1111 110 & L & Global ID & Subnet ID & Interface ID \\
            \hline
        \end{tabular}
    \end{center}
\end{table}

Lokální síť je tedy složená ze tří části:
\begin{itemize}
    \item unikátní prefix délky 48 bitů,
    \item 16 bitový identifikátor podsítě,
    \item adresa uzlu o~délce 64 bitů.
\end{itemize}

\documentclass[a4paper,11pt]{article}

\usepackage[utf8]{inputenc}
\usepackage[czech]{babel}
\usepackage[left=2cm,top=3cm,text={17cm,24cm}]{geometry}
\usepackage{graphicx}
\usepackage{listings}
\usepackage{url}

\title{Základní konfigurace síťových zařízení a analýza síťového provozu programem Wireshark\\
{\bf\large ISA - Laboratorní cvičení č.1}\\
{\bf\large Protokol ke cvičení}}

\author{Vysoké učení technické v Brně}

\date{\url{https://github.com/nesfit/ISA/tree/master/lab1-konfigurace}}

\setlength\parindent{0pt}

\begin{document}

{\let\newpage\relax\maketitle}

%Jméno:\\
Login:\\

\section{Zjišťování konfigurace}
\textbf{1.}\\
Rozhraní eth0\\
\\
\begin{tabular}{|l|r|}
\hline
MAC adresa: & \hspace{12em} \\
\hline
IPv4 adresa: & \hspace{12em} \\
\hline
Délka prefixu: & \\
\hline
Adresa sítě: & \\
\hline
Broadcastová adresa: & \\
\hline
\end{tabular}
\vspace{1.5em}
\\
\textbf{2.}\\
Záznam výchozí brány ve směrovací tabulce:\\
\vskip 3em
IPv4 adresa:\\
MAC adresa:\\
\\
\textbf{4.}\\
Soubor:\\
Implicitní DNS servery:\\
\vskip 4em
\textbf{5.}\\
Soubor:\\
Úprava:\\
\vskip 4em
\textbf{6.}\\
Záznam + popis:\\
\vskip 8em

\section{Wireshark}
\textbf{1.}\\
Capture filter:\\
\\
\textbf{2.}\\
\\
\begin{tabular}{|l|c|c|}
\hline
\textbf{Hodnota} & \textbf{požadavek} & \textbf{odpověď}\\
\hline
Cílová MAC adresa & \hspace{10em} & \hspace{10em} \\
\hline
Cílová IPv4 adresa & & \\
\hline
Zdrojová MAC adresa & & \\
\hline
Zdrojová IPv4 adresa & & \\
\hline
\end{tabular}
\vspace{2em}
\\
Komu patří nalezené IPv4 adresy a MAC adresy? Vypisovali jste již některé z nich? Proč tomu tak je?\\
\vskip 6em
\textbf{3.}\\
Filtr:\\
\vskip 4em
\textbf{6.}\\
Jaký je formát zobrazených dat funkcí \emph{Follow TCP stream}, slovně popište
význam funkce \emph{Follow TCP stream}\\


%\thispagestyle{empty}

Jméno a příjmení:\\
Login:\\
Skupina (číslo nebo čas):\\
Datum:\\

\section{Zjišťování konfigurace}
\textbf{1.1}
Konfigurace rozhraní \texttt{enp2s0}\\
\\
\begin{tabular}{|l|r|}
\hline
MAC adresa: & \hspace{20em} \\
\hline
IPv4 adresa: & \\
\hline
Délka prefixu (v bitech): & \\
\hline
Adresa sítě: & \\
\hline
Broadcastová adresa: & \\
\hline
\end{tabular}
\vspace{1em}
\\
\textbf{1.2} Výchozí brána (default gateway)
\medskip

IPv4 adresa:\\
MAC adresa:\\
\\
\textbf{1.4} Implicitní DNS servery:\\
\vskip 2em

\textbf{1.5} Přidaný záznam v {\tt /etc/hosts}:\\
\vskip 2em

\textbf{1.6} Příklad aktivního spojení + vysvětlení položek:\\
\vskip 8em

\textbf{1.8} Vyhledání chybové zprávy v systémovém logu
\medskip

Příkaz použitý na zobrazení chyby:\\

Příčina chyby (vysvětlete):

\section{Wireshark}
\textbf{2.1} Capture filter:\\
\vskip 1em

\textbf{2.5} Analýza komunikace HTTP\\

\begin{tabular}{|l|c|c|c|c|}
\hline
& \multicolumn{2}{|c|}{\textbf{Požadavek HTTP}} & \multicolumn{2}{|c|}{\textbf{Odpověď HTTP}}\\
\hline
\textbf{Hodnota} & \textbf{Adresa} & \textbf{Typ zařízení} & \textbf{Adresa} & \textbf{Typ zařízení}\\
\hline
Zdrojová MAC adresa & & & & \\
\hline
Zdrojová IPv4 adresa & & & & \\
\hline
Cílová MAC adresa & \hspace{8em} & \hspace{8em} & \hspace{8em} & \hspace{8em} \\
\hline
Cílová IPv4 adresa & & & & \\
\hline
\end{tabular}

\bigskip
U typu zařízení uveďte, jakou funkci má dané zařízení, např. webový klient či server, brána apod. 

\bigskip

\textbf{2.6} Popište rozdíl výstupů funkce {\em Follow TCP stream} a {\em Follow HTTP stream}. \\
\medskip

\vskip 4em

\textbf{2.10} Nastavení filtru pro zobrazení (display filter) v programu Wireshark:\\
\medskip

\textbf{2.11} Popište souvislost odchycených paketů DNS a HTTP(S):\\
\vskip 5em

\section{Konfigurace IPv4 a IPv6}

\begin{tabular}{|l|r|}
\hline
Zvolená maska podsítě pro IPv4: & \hspace{20em} \\ \hline
Použitá adresa IPv4: & \\ \hline
Maximální počet zařízení v podsíti: & \\ \hline 
\end{tabular}

\bigskip
\begin{tabular}{|l|r|}
\hline
Zvolený prefix IPv6 sítě: & \hspace{20em} \\ \hline
Použitá adresa IPv6: & \\ \hline
Maximální počet zařízení v podsíti: & \\ \hline
\end{tabular}
\end{document}
